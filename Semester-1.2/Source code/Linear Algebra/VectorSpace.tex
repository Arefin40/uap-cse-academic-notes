\documentclass[12pt]{article}
\usepackage[margin=1.27cm]{geometry}
\usepackage{setspace}
\usepackage{fontspec}
% \usepackage[T1]{fontenc}
% \usepackage[utf8]{inputenc}
\usepackage{amsmath,txfonts,amssymb,nicefrac,mathtools,pifont} %for math
\usepackage{array,tabularx,multirow,fmtcount} %for tables
\usepackage{tikz, pgfplots} %for diagram
\usepackage{multicol} %for multiple column
\usepackage{enumerate,enumitem,adjustbox} %for ordered list
\usepackage{graphicx,subcaption,wrapfig,tcolorbox} %for figure
\usepackage{xparse} %for commands & environments
\usepackage{lipsum} %miscellaneous
\usepackage{colortbl,xcolor,soul} %for default table & border

% #ANCHOR Font settings
\setmainfont{Oxygen}
\newfontfamily\banglafont[Script=Bengali]{Baloo Da 2}
\newfontfamily{\lstsansserif}{IBM Plex Mono}
\renewcommand{\normalsize}{\fontsize{11.5pt}{13pt}\selectfont}


\setlength{\arrayrulewidth}{0.35 pt}
\definecolor{border}{HTML}{A1A1AA}
\arrayrulecolor{border}


% #ANCHOR Document settings
\linespread{1.45}
\setlength\parindent{0pt}
\setlength\parskip{16pt}
\setlist[enumerate]{noitemsep}
\usetikzlibrary{shapes.geometric,decorations.pathreplacing,trees,arrows,positioning,shapes,fit,calc,decorations.markings, decorations.text}
\tikzset{every node/.append style={font=\footnotesize}}
\usepgfplotslibrary{fillbetween}
\pgfdeclarelayer{background}
\pgfsetlayers{background,main}
\pgfplotsset{compat=1.18}
\columnseprule=1pt
\everymath{\displaystyle}
% #ANCHOR Hypernation
\tolerance=1
\emergencystretch=\maxdimen
\hyphenpenalty=10000
\hbadness=10000
\newlength{\colWidth}



% #ANCHOR Colors
\definecolor{azure(colorwheel)}{rgb}{0.0, 0.5, 1.0}
\definecolor{carminepink}{rgb}{0.92, 0.3, 0.26}
\definecolor{orange}{rgb}{0.9, 0.55, 0.22}
\definecolor{violet}{rgb}{0.60, 0.45, 1}
% Syantax Highlighting Colors
\definecolor{keyword}{HTML}{D73A4A}
\definecolor{number}{HTML}{015CC5}
\definecolor{comment}{HTML}{6A737D}
\definecolor{string}{HTML}{1D825E}
\definecolor{function}{HTML}{743FD1}
\definecolor{orange}{HTML}{CF7842}
\definecolor{codeblack}{HTML}{24292F}
\definecolor{divider}{HTML}{A1A1AA}
\definecolor{border}{HTML}{D1D1D1}


% #ANCHOR Ordered & Unordered List
\setlist[itemize,1]{left=0cm, label={\textbullet}}
\setlist[itemize,2,3,4,5,6,7,8,9,10]{left=0.6cm, label={\textbullet}}
\setlist[enumerate,1]{left=0cm}
\setlist[enumerate,2,3,4,5,6,7,8,9,10]{left=0.6cm}
\setul{0.5ex}{0.125ex}



% #ANCHOR Colored Box
\let\oldul\ul
\renewcommand{\ul}[2][keyword]{\text{\setulcolor{#1}\oldul{#2}}}
\newcommand{\redbox}[1]{%
{\color{red}\fbox{\color{black}#1}}
}
\newcommand{\red}[1]{%
\textcolor{red}{#1}
}
\newcommand{\redeq}[1]{%
\text{\color{red}$#1$}
}
\newcommand{\mred}[1]{%
\textcolor{keyword}{#1}
}
\newcommand{\mredeq}[1]{%
\textcolor{keyword}{$#1$}
}
\newcommand{\blue}[1]{%
% {\color{number}#1\hspace{-0.4ex}}
\textcolor{number}{#1}
}
\newcommand{\blueeq}[1]{%
\text{\color{number}$#1$}
}
\newcommand{\cyanbox}[1]{%
{\color{teal}\fbox{\textcolor{black}{#1}}}
}
\newcommand{\cyan}[1]{%
\textcolor{teal}{#1}
}
\newcommand{\pink}[1]{%
\textcolor{magenta}{#1}
}
\newcommand{\orange}[1]{%
\textcolor{orange}{#1}
}
\newcommand{\violet}[1]{%
{\color{violet}#1}
}
\newcommand{\cyaneq}[1]{%
\text{\color{teal}$#1$}
}
\newcommand{\gray}[1]{%
\textcolor{comment}{#1}
}
\newcommand{\pinkeq}[1]{%
\text{\color{magenta}$#1$}
}
\renewcommand{\columnseprulecolor}{\color{divider}}




% #ANCHOR Tabular commands
\newcolumntype{P}[1]{>{\centering\arraybackslash}p{#1}}
\newcolumntype{M}[1]{>{\centering\arraybackslash}m{#1}}
\newcolumntype{C}{>{\centering\arraybackslash}X}
\newcommand{\rspan}[2]{\multirow{#1}{*}{#2}}
\newcommand{\thc}[1]{%
\multicolumn{1}{|c|}{\textbf{#1}}
}
\newcommand{\thcx}[1]{%
\multicolumn{1}{|C|}{\textbf{#1}}
}
\newcommand{\thl}[1]{%
\multicolumn{1}{|l|}{\textbf{#1}}
}
\newcommand{\thr}[1]{%
\multicolumn{1}{|r|}{\textbf{#1}}
}
% Adjusting arraystretch to modify vertical padding
\renewcommand{\arraystretch}{1.25}
% Adjusting tabcolsep to modify horizontal padding
\setlength{\tabcolsep}{10pt}



% #ANCHOR Math commands
\newcommand{\set}[1]{\{$#1$\}}
\newcommand{\tabs}{\ \ \ \ \ \ }
\newcommand{\tab}{\ \ \ }
\newcommand{\cmark}{\ding{51}}%
\newcommand{\xmark}{\ding{55}}%
\newcommand{\boldi}[1]{\boldsymbol{#1}}%
\newcommand{\wspace}{\ \ = \ \ }



% #ANCHOR New commands
\newcommand{\Title}[1]{%
   \begin{center}
      \textbf{\Large{#1}}
   \end{center}
}
\newcommand{\Heading}[1]{%
   \par\vspace{\dimexpr -\baselineskip + 16pt}
   {\fontsize{12pt}{13pt}\selectfont\textbf{#1}}
   \par\vspace{\dimexpr -\baselineskip + 6pt}
}
\newcommand{\BuleHeading}[1]{%
   \par\vspace{\dimexpr -\baselineskip + 16pt}
   {\fontsize{12pt}{13pt}\selectfont\textbf{\textcolor{number}{#1}}}
   \par\vspace{\dimexpr -\baselineskip + 6pt}
}
\newcommand{\CHeading}[1]{%
   \par\vspace{\dimexpr -\baselineskip + 16pt}
   \hspace{\fill}
   {\fontsize{12pt}{13pt}\selectfont\textbf{#1}}
   \hspace{\fill}
   \par\vspace{\dimexpr -\baselineskip + 6pt}
}
\newcommand{\Section}[1]{%
   \par\vspace{\dimexpr -\baselineskip + 16pt}
   \hspace{\fill}
   {\fontsize{13pt}{13pt}\selectfont\textbf{#1}}
   \hspace{\fill}
   \par\vspace{\dimexpr -\baselineskip + 6pt}
}
\newcommand{\seteqno}[1]{%
   \ \cdots \ \cdots \ \cdots \ (#1)
}
\newcommand{\eqor}{%
   \Rightarrow \ \ 
}
\newcommand{\tsub}[1]{%
\textsubscript{#1}\hspace{-0.45ex}
}
\newcommand{\tsup}[1]{%
\textsuperscript{#1}\hspace{-0.45ex}
}
\newcommand{\cbox}[2][cyan]{
\tikz\node[draw=#1,circle,inner sep=2pt,baseline=(a.base)](a){#2};
}
\newcommand{\hrline}{%
\vspace{1ex} {\color{gray}\hrule} \vspace{4ex}
}
\newcommand{\divideX}[1][divider]{{\hspace{1ex}\color{#1}{\vrule}\hspace{1ex}}}
\newcommand{\Reference}[2][Reference]{

\vspace{-0.5\baselineskip}
\begin{center}
   {\fontspec{Merriweather}\textbf{#1:} \textit{#2}} 
\end{center}
}
\newcommand{\bn}[1]{%
   {\banglafont #1}
}

\NewDocumentCommand{\Column}{O{0.49} O{1.5em} m m}{
   \setlength{\colWidth}{\linewidth-#1\linewidth-#2}
   \begin{minipage}[t]{#1\linewidth}
      \noindent
         #3
      \end{minipage}\hspace{\fill}{\color{divider}\vrule width 0.35pt}\hspace{\fill}
      \begin{minipage}[t]{\colWidth}
      \noindent
         #4
   \end{minipage}
}

% New commands
\newcommand{\ro}[2][]{
\tab \xrightarrow[\scalebox{1}{$#1$}]{\scalebox{1}{$#2$}} \tab
}
\renewcommand{\vec}[1]{\overrightarrow{#1}}

\renewenvironment{pmatrix}
{\left(\begin{array}{ccc}}
{\end{array}\right)}

\NewDocumentEnvironment{fmatrix}{m}
{\left(\begin{array}{#1}}
{\end{array}\right)}

\NewDocumentEnvironment{pomatrix}{mm}
{\left(\begin{array}{ccc}}
{\end{array}\right)_{#1 \times #2}}

\NewDocumentEnvironment{figmatrix}{moO{0.3}}
{  
   \begin{subfigure}{#3\textwidth} \centering $$
      \begin{fmatrix}{#1}
}
{
      \end{fmatrix} $$ \IfValueT{#2}{#2}
   \end{subfigure}
}
\NewDocumentEnvironment{csplit}{o}
{  
   \IfValueT{#1}{\begin{center}}
      \begin{tabular}{l|l}
}
{
      \end{tabular}
   \IfValueT{#1}{\end{center}}
}
\newcommand{\sxrightarrow}[2][]{%
  \mathrel{\text{$\xrightarrow[#1]{#2}$}}%
}

   


\begin{document}

\Lecture{Linear combination}

% \begin{csplit}[c]
%    $\vec{u} \ =\ a_1\hat{\ i}+b_1\hat{\ j}+c_1\hat{\ k} \ =\ (a_1,b_1,c_1)$ \\
%    $\vec{v} \ =\ a_2\hat{\ i}+b_2\hat{\ j}+c_2\hat{\ k} \ =\ (a_2,b_2,c_2)$
%    \tabs & \tabs
%    $\vec{u} \ =\ a_1a_2 \ +\ b_1b_2 \ +\ c_1c_2$\\
%    $A= \left| \ \ \begin{matrix}
%       \hat{i} & \hat{j}  & \hat{k} \\
%       a_1 & b_1 & c_1 \\
%       a_2 & b_2 & c_2 
%    \end{matrix} \ \ \right|$
% \end{csplit}



\Heading{Linear combination: \cyan{vector}}

\textbf{Linear Combination of vectors:} Let \ $V$ \ be a vector space over the field \ $F$ \ and let \ $\mathrm{v}_1, \mathrm{v}_2, \ldots \ldots \ldots, \mathrm{v}_{\mathrm{n}} \in \mathrm{V}$. Then any vector \ $\mathrm{v} \in \mathrm{V}$ \ is called a linear combination of \ $\mathrm{v}_1, \mathrm{v}_2, \ldots \ldots \ldots, \mathrm{v}_{\mathrm{n}}$ \ if and only if there exists scalar \ $\alpha_1, \alpha_2, \ldots \ldots \ldots, \alpha_n$ \ in \ $F$ \ such that, \ $\mathrm{v}=\alpha_1 \mathrm{v}_1+\alpha_2 \mathrm{v}_2+\cdots \ldots \ldots+\alpha_n \mathrm{v}_{\mathrm{n}}$

\vspace{5ex}
\cyanbox{Example 1:} Consider the vectors $\mathbf{u}=(1,2,-1)$ and $\mathbf{v}=(6,4,2)$ in $\mathbb{R}^3$. Show that $\mathbf{w}=(9,2,7)$ is a linear combination of $\mathbf{u}$ and $\mathbf{v}$.


\vspace{3ex}
\cyanbox{Solution:}
\vspace{-3\baselineskip}

\begin{align*}
   \text{Let, \ \ } w \ &=\ \alpha_1 u \ +\ \alpha_2 v \tabs \alpha_i \ \in \mathbb{R}\\
   \Rightarrow \tab (9,2,7) \ &=\ \alpha_1 (1,2,-1) \ +\ \alpha_2 (6,4,2) \\
   \Rightarrow \tab (9,2,7) \ &=\ (\alpha_1,2\alpha_1,-\alpha_1) \ +\ (6\alpha_2,4\alpha_2,2\alpha_2) \\
   \ &=\ (\alpha_1+6\alpha_2, \tab 2\alpha_1+4\alpha_2, \tab -\alpha_1+2\alpha_2) 
\end{align*}

Equating corresponding components:
\vspace{-0.75\baselineskip}
\begin{equation*}
   \left.
   \begin{array}{r}
   \alpha_1 \ + \ 6 \alpha_2\ = \ 9 \\
   2 \alpha_1 \ + \ 4 \alpha_2\ = \ 2 \\
   -\alpha_1 \ + \ 2 \alpha_2\ = \ 7
   \end{array}
   \tab \right\} \tab \seteqno{1}
\end{equation*}


Writing $(1)$ in matrix form $A \mathbf{x}=\mathbf{b}$,
\begin{equation*}
   A=\begin{fmatrix}{cc}
      1 & 6 \\ 2 & 4 \\ -1 & 2
   \end{fmatrix}, \quad
   \mathbf{x} \ = \ \begin{fmatrix}{l}
      \alpha_1 \\ \alpha_2
   \end{fmatrix}, \quad
   \mathbf{b} \ = \ \begin{fmatrix}{r}
      9 \\ 2 \\ 7
   \end{fmatrix}
\end{equation*}


Reduce augmented matrix \ $(A \mid \mathbf{b})$ \ to echelon form by e.r.o
\begin{align*}
   \left(A \mid \mathbf{b}\right) = \ 
   \begin{fmatrix}{cc|c}
      1 & 6 & 9 \\ 2 & 4 & 2 \\ -1 & 2 & 7
   \end{fmatrix}
      \ro{\tabs}
   \begin{fmatrix}{cc|c}
      1 & 6 & 9 \\ 0 & -8 & -16 \\ 0 & 8 & 16
   \end{fmatrix}
      \ro{\tabs}
   \begin{fmatrix}{cc|c}
      1 & 6 & 9 \\ 0 & -8 & -16 \\ 0 & 0 & 0
   \end{fmatrix}
      \ro{\tabs}
   \begin{fmatrix}{cc|c}
      1 & 6 & 9 \\ 0 & 1 & 2 \\ 0 & 0 & 0
   \end{fmatrix}
\end{align*}

Corresponding system,
\vspace{-\baselineskip}
\begin{alignat*}{4}
   \alpha_1 &\ +\ & 6 \alpha_2 &\ =\ & 9 \\
   && \alpha_2 &\ =\ & 2 \\
\end{alignat*}

\vspace{-\baselineskip}
By backward substitution,
$$\alpha_1 = -3 \tabs \text{and} \tabs \mathbf{x} \ = \ \begin{fmatrix}{c}
   -3\\2
\end{fmatrix}$$
\vspace{-\baselineskip}
\begin{center}
$\therefore w \ = -3u \ + \ 2v$

Therefore, \ $w$ \ is a linear combination of \ $u$ \ and \ $v$
\end{center}


\vspace{5ex}
\cyanbox{Example 2:} Is the vector $v=(2,-5,3)$ in $\mathbb{R}^3$ is a linear combination of the vectors $v_1=(1,-3,2), \\ v_2=(2,-4,-1)$ and $v_3=(1,-5,7)$

\vspace{3ex}
\cyanbox{Solution:}
\vspace{-\baselineskip}
\begin{align*}
   \text{Let, \ \ } v \ &=\ \alpha_1 v_1 \ +\ \alpha_2 v_2 \ +\ \alpha_2 v_3 \tabs \alpha_i \ \in \mathbb{R}\\
   \Rightarrow \tab (2,-5,3) \ &=\ \alpha_1 (1,-3,2) \ +\ \alpha_2 (2,-4,-1) \ +\ \alpha_2 (1,-5,7)\\
   \Rightarrow \tab (2,-5,3) \ &=\ (\alpha_1,-3 \alpha_1, 2 \alpha_1) \ +\ (2 \alpha_2,-4 \alpha_2,-\alpha_2) \ +\ (\alpha_3,-5 \alpha_3, 7 \alpha_3) \\
   \ &=\ (\alpha_1+2 \alpha_2+\alpha_3, \tab -3 \alpha_1-4 \alpha_2-5 \alpha_3, \tab 2 \alpha_1-\alpha_2+7 \alpha_3) 
\end{align*}

Equating corresponding components:
\vspace{-0.75\baselineskip}
\begin{equation*}
   \left.
   \begin{array}{r}
      \alpha_1 \ + \ 2 \alpha_2 \ + \ \alpha_3 \ = \ 2 \\
      -3 \alpha_1 \ - \ 4 \alpha_2 \ - \ 5 \alpha_3 \ = \ -5 \\
      2 \alpha_1 \ - \ \alpha_2 \ + \ 7 \alpha_3 \ = \ 3
   \end{array}
   \tab \right\} \tab \seteqno{1}
\end{equation*}


Writing $(1)$ in matrix form $A \mathbf{x}=\mathbf{b}$,
\begin{equation*}
   A=\begin{fmatrix}{rrr}
      1 & 2 & 1 \\ -3 & -4 & -5 \\ 2 & -1 & 7
   \end{fmatrix}, \quad
   \mathbf{x} \ = \ \begin{fmatrix}{l}
      \alpha_1 \\ \alpha_2 \\ \alpha_3
   \end{fmatrix}, \quad
   \mathbf{b} \ = \ \begin{fmatrix}{r}
      2 \\ -5 \\ 3
   \end{fmatrix}
\end{equation*}


Reduce augmented matrix \ $(A \mid \mathbf{b})$ \ to echelon form by e.r.o
\begin{align*}
   \left(A \mid \mathbf{b}\right) = \ 
   \begin{fmatrix}{rrr|r}
      1 & 2 & 1 & 2 \\ -3 & -4 & -5 & -5 \\ 2 & -1 & 7 & 3
   \end{fmatrix}
      \ro{\tabs}
   \begin{fmatrix}{rrr|r}
      1 & 2 & 1 & 2 \\ 0 & 2 & -2 & 1 \\ 0 & -5 & 5 & -1
   \end{fmatrix}
      \ro{\tabs}
   \begin{fmatrix}{rrr|r}
      1 & 2 & 1 & 2 \\ 0 & 2 & -2 & 1 \\ 0 & 0 & 0 & 3
   \end{fmatrix}
\end{align*}

Corresponding system,
\vspace{-\baselineskip}
\begin{alignat*}{4}
   \alpha_1 &\ +\ & 2 \alpha_2 &\ +\ & \alpha_3 &\ =\ & 2 \\
   && 2 \alpha_2 &\ -\ & 2\alpha_3 &\ =\ & 1 \\
   &&&& 0 &\ =\ & 3 \\
\end{alignat*}

\vspace{-\baselineskip}
There is an equation in the form \ $0=b$ \ and \ $b\neq 0$

Hence the above system is inconsistent and it has no solution.

Therefore, the vector \ $v$ \ is not a linear combination of \ $v_1$, \ $v_2$ \ and \ $v_3$ \ .

\vspace{5ex}
\Heading{Linear combination: \cyan{Matrix}}
\vspace{1ex}
\cyanbox{Example:} Write the matrix $A=\begin{pmatrix} 3 & 1 \\[-1ex] 1 & -1 \end{pmatrix}$ as a linear combination of the matrices $A_1=\begin{pmatrix} 1 & 1 \\[-1ex] 1 & 0 \end{pmatrix}$, \ $A_2=\begin{pmatrix} 0 & 0 \\[-1ex] 1 & 1 \end{pmatrix}$, \ and $A_2=\begin{pmatrix} 0 & 2 \\[-1ex] 0 & -1 \end{pmatrix}$


\vspace{3ex}
\cyanbox{Solution:}
\vspace{-2\baselineskip}
\begin{align*}
   \text{Let, \ } S \ &=\ \{A_1, \ A_2, \ A_3\} \tabs \text{and} \tabs M=\begin{pmatrix} 3 & 1 \\[-1ex] 1 & -1 \end{pmatrix}\\
   \text{Set, \ \ } M \ &=\ \alpha_1 A_1 \ +\ \alpha_2 A_2 \ +\ \alpha_3 A_3\\[1ex]
   \Rightarrow \tab
   \begin{pmatrix} 3 & 1 \\[-1ex] 1 & -1 \end{pmatrix}
      \ &=\ \alpha_1
   \begin{pmatrix} 1 & 1 \\[-1ex] 1 & 0 \end{pmatrix}
      \ +\ \alpha_2 
   \begin{pmatrix} 0 & 0 \\[-1ex] 1 & 1 \end{pmatrix}
      \ +\ \alpha_3 
   \begin{pmatrix} 0 & 2 \\[-1ex] 0 & -1 \end{pmatrix}\\[1ex]
      \Rightarrow \tab
   \begin{pmatrix} 3 & 1 \\[-1ex] 1 & -1 \end{pmatrix} \ &=\ \begin{pmatrix}
      \alpha_1 & \alpha_1 \\[-1ex] \alpha_1 & 0
   \end{pmatrix} \ +\ \begin{pmatrix}
     0 & 0 \\[-1ex] \alpha_2 & \alpha_2
   \end{pmatrix} \ +\ \begin{pmatrix}
      0 & 2\alpha_3 \\[-1ex] 0 & -\alpha_3
   \end{pmatrix} \\
   \ &=\
   \begin{pmatrix}
      \alpha_1 & \alpha_1+2 \alpha_3 \\
      \alpha_1+\alpha_2 & \alpha_2-\alpha_3
   \end{pmatrix} 
\end{align*}

Equating corresponding components:
\vspace{-0.75\baselineskip}
\begin{equation*}
   \left.
   \begin{array}{rl}
      \alpha_1 &=\ 3 \\
      \alpha_1+2 \alpha_3 &=\ 1 \\
      \alpha_1+\alpha_2 &=\ 1 \\
      \alpha_2-\alpha_3 &=\ -1
   \end{array}
   \tab \right\} \tab \seteqno{1}
\end{equation*}


Writing $(1)$ in matrix form $A \mathbf{x}=\mathbf{b}$,
\begin{equation*}
   A=\begin{fmatrix}{rrr}
      1 & 0 & 0  \\ 1 & 0 & 2 \\ 1 & 1 & 0 \\ 0 & 1 & -1
   \end{fmatrix}, \quad
   \mathbf{x} \ = \ \begin{fmatrix}{l}
      \alpha_1 \\ \alpha_2 \\ \alpha_3
   \end{fmatrix}, \quad
   \mathbf{b} \ = \ \begin{fmatrix}{r}
      3 \\ 1 \\ 1 \\ -1
   \end{fmatrix}
\end{equation*}


Reduce augmented matrix \ $(A \mid \mathbf{b})$ \ to echelon form by e.r.o
\begin{align*}
   \left(A \mid \mathbf{b}\right) = \ 
   &\begin{fmatrix}{rrr|r}
      1 & 0 & 0 & 3 \\
      1 & 0 & 2 & 1 \\
      1 & 1 & 0 & 1 \\
      0 & 1 & -1 & -1\\
   \end{fmatrix}
      \ro[R_4 \leftrightarrow R_2]{R_3 \leftrightarrow R_1}
   \begin{fmatrix}{rrr|r}
      1 & 1 & 0 & 1 \\
      0 & 1 & -1 & -1\\
      1 & 0 & 0 & 3 \\
      1 & 0 & 2 & 1 \\
   \end{fmatrix}
      \ro[{R_4'= R_4-R_1}]{R_3'= R_3-R_1}
   \begin{fmatrix}{rrr|r}
      1 & 1 & 0 & 1 \\
      0 & 1 & -1 & -1\\
      0 & -1 & 0 & 2 \\
      0 & -1 & 2 & 0 \\
   \end{fmatrix}\\[1ex]
   \ro[{R_4'= R_4+R_2}]{R_3'= R_3+R_2}
   &\begin{fmatrix}{rrr|r}
      1 & 1 & 0 & 1 \\
      0 & 1 & -1 & -1\\
      0 & 0 & -1 & 1 \\
      0 & 0 & 1 & -1 \\
   \end{fmatrix}
      \ro{R_4'= R_4+R_3}
   \begin{fmatrix}{rrr|r}
      1 & 1 & 0 & 1 \\
      0 & 1 & -1 & -1\\
      0 & 0 & -1 & 1 \\
      0 & 0 & 0 & 0 \\
   \end{fmatrix}
\end{align*}

Corresponding system,
\vspace{-\baselineskip}
\begin{alignat*}{4}
   \alpha_1 &\ +\ & \alpha_2 &&&\ =\ & 1 \\
   && \alpha_2 &\ -\ & \alpha_3 &\ =\ & -1 \\
   &&&\ -\ & \alpha_3 &\ =\ & 1
\end{alignat*}

\vspace{-\baselineskip}
By backward substitution,
$$\begin{aligned}\alpha_3 &= -1\\\alpha_2 &= -2\\\alpha_1 &= 3\end{aligned} \tabs \text{and} \tabs \mathbf{x} \ = \ \begin{fmatrix}{c}
   3\\-2\\-1
\end{fmatrix}$$
\vspace{-\baselineskip}
\begin{center}
$\therefore A \ = 3A_1-2A_2-A_3$

Therefore, the matrix \ $A$ \ is a linear combination of \ $A1$, \ $A_2$ \ and \ $A_3$
\end{center}

\vspace{5ex}
\Heading{Spanning: \cyan{Polynomial}}
\vspace{1ex}
\cyanbox{Example:} Determine whether the polynomials:

\begin{equation*}
   \left.
   \begin{aligned}
   & P_1=1-x+2x^2 \\
   & P_2=3+x \\
   & P_3=5-x+4x^2 \\
   & P_4=-2-2 x+2x^2 
   \end{aligned} \tab \right\} \ \text{spans} \ \ \mathbb{P}_2
\end{equation*}

\vspace{3ex}
\cyanbox{Solution:} Let $P \in \mathbb{P}_2$ be anbitary, then:
\vspace{-0.75\baselineskip}
\begin{align*}
P \ &= \ a+bx+cx^2 . \\
\text {Set, \ } P \ &= \ \alpha_1 P_1+\alpha_2 P_2+\alpha_3 P_3+\alpha_4 P_4 \\
a+bx+cx^2 \ &= \ \alpha_1(1-x+2x^2) \ + \ \alpha_2(3+x)
\ + \ \alpha_3(5-x+4x^2) \ + \ \alpha_4(-2-2 x+2x^2)\\
\ &= \ (\alpha_1+3 \alpha_2+5 \alpha_3-2 \alpha_4)+(-\alpha_1+\alpha_2-\alpha_3-2 \alpha_4) x+(2 \alpha_1+0+4 \alpha_3+2\alpha_4) x^2
\end{align*}

Equating corresponding components:
\vspace{-0.75\baselineskip}
\begin{align*}
   \alpha_1+3 \alpha_2+5 \alpha_3-2 \alpha_4 \ &= \ a\\
   -\alpha_1+\alpha_2-\alpha_3-2 \alpha_4 \ &= \ b\\
   2\alpha_1+0+4 \alpha_3+2\alpha_4 \ &= \ c
\end{align*}


Writing $(1)$ in matrix form $A \mathbf{x}=\mathbf{b}$,
\begin{equation*}
   A=\begin{fmatrix}{rrrr}
      1 & 3 & 5 & -2  \\ -1 & 1 & -1 & -2 \\ 2 & 0 & 4 & 2
   \end{fmatrix}, \quad
   \mathbf{x} \ = \ \begin{fmatrix}{l}
      \alpha_1 \\ \alpha_2 \\ \alpha_3 \\ \alpha_4
   \end{fmatrix}, \quad
   \mathbf{b} \ = \ \begin{fmatrix}{r}
      a \\ b \\ c
   \end{fmatrix}
\end{equation*}


Reduce augmented matrix \ $(A \mid \mathbf{b})$ \ to echelon form by e.r.o
\begin{align*}
   \left(A \mid \mathbf{b}\right) = \ 
   &\begin{fmatrix}{rrrr|r}
      1 & 3 & 5 & -2 & a \\
      -1 & 1 & -1 & -2 & b \\
      2 & 0 & 4 & 2 & c
   \end{fmatrix}
      \ro{R_2'=R_1+R_2}
   \begin{fmatrix}{rrrr|l}
      1 & 3 & 5 & -2 & a \\
      0 & 4 & 4 & -4 & a+b \\
      0 & -6 & -6 & 6 & c-2a
   \end{fmatrix}\\[1ex]
      \ro{R_3'=4R_3+6R_2}
   &\begin{fmatrix}{rrrr|l}
      1 & 3 & 5 & -2 & a \\
      0 & 4 & 4 & -4 & a+b \\
      0 & 0 & 0 & 0 & 6c-8a+4
   \end{fmatrix}
\end{align*}

Corresponding system,
\vspace{-\baselineskip}
\begin{alignat*}{12}
   \alpha_1 &\ +\ & 3\alpha_2 &\ +\ & 5\alpha_3 &\ -\ & 2\alpha_4 &\ =\ & a \\
   && 4\alpha_2 &\ +\ & 4\alpha_3 &\ -\ & 4\alpha_4 &\ =\ & a+b \\
   &&&&&& 0 &\ =\ & 6c-8a+4b \\
\end{alignat*}


\vspace{-\baselineskip}
There is an equation in the form \ $0=b$ \ and \ $b\neq 0$

Hence the above system is inconsistent and it has no solution.

Therefore, the polynomials doesn't not spans \ $\mathbb{P}_2$






\vspace{5ex}
\Lecture{Linear Independence: \cyan{vector}}
\cyanbox{Example:} Determine wheather the vactors \ \ $v_1=(1,-2,3), \ \ v_2=(5,6,-1), \ \ v_{3}=(3,2,1)$ \ \ are Linearly Dependent or Independent in \ $\mathbb{R}^3$

\vspace{3ex}
\cyanbox{Solution:}
\vspace{-3\baselineskip}

\begin{align*}
   \text{Let, \ \ } \alpha_1 v_1 \ +\ \alpha_2 v_2 \ +\ \alpha_3 v_3 \ & \ = \ 0 \tabs \alpha_i \ \in \mathbb{R}\\
   \alpha_1 (1,-2,3) \ +\ \alpha_2 (5,6,-1) \ +\ \alpha_3 (3,2,1) \ & \ = \ (0,0,0)  \\
   (\alpha_1,-2\alpha_1,3\alpha_1) \ +\ (5\alpha_2,6\alpha_2,-\alpha_2) \ +\ (3\alpha_3,2\alpha_3,\alpha_3) \ & \ = \ (0,0,0)  \\
   (\alpha_1+5\alpha_2+3\alpha_3, \tab
-2\alpha_1+6\alpha_2+2\alpha_3, \tab
3\alpha_1-\alpha_2+\alpha_3) \ & \ = \ (0,0,0)
\end{align*}

Equating corresponding components:
\vspace{-0.75\baselineskip}
\begin{equation*}
   \left.
   \begin{aligned}
      \alpha_1+5\alpha_2+3\alpha_3 &= 0 \\
      -2\alpha_1+6\alpha_2+2\alpha_3\ &= \ 0 \\
      3\alpha_1-\alpha_2+\alpha_3\ &= \ 0
   \end{aligned}
   \tab \right\} \tab \seteqno{1}
\end{equation*}


Writing $(1)$ in matrix form $A \mathbf{x}=\mathbf{b}$,
\begin{equation*}
   A=\begin{fmatrix}{rrr}
      1 & 5 & 3 \\ -2 & 6 & 2 \\ 3 & -1 & 1
   \end{fmatrix}, \quad
   \mathbf{x} \ = \ \begin{fmatrix}{l}
      \alpha_1 \\ \alpha_2 \\ \alpha_3
   \end{fmatrix}, \quad
   \mathbf{b} \ = \ \begin{fmatrix}{r}
      0 \\ 0 \\ 0
   \end{fmatrix}
\end{equation*}


Reduce augmented matrix \ $(A \mid \mathbf{b})$ \ to echelon form by e.r.o
\begin{align*}
   \left(A \mid \mathbf{b}\right) = \ 
   \begin{fmatrix}{rrr|l}
      1 & 5 & 3 & 0 \\ -2 & 6 & 2 & 0 \\ 3 & -1 & 1 & 0
   \end{fmatrix}
      \ro[R_3'=R_3-3R_1]{R_2'=R_2+2R_1}
   \begin{fmatrix}{rrr|l}
      1 & 5 & 3 & 0 \\ 0 & 16 & 8 & 0 \\ 0 & -16 & -8 & 0
   \end{fmatrix}
   \ro[R_3'=R_3+R_2]{R_2'=\frac{R_2}{8}}
   \begin{fmatrix}{rrr|l}
      1 & 5 & 3 & 0 \\ 0 & 2 & 1 & 0 \\ 0 & 0 & 0 & 0
   \end{fmatrix}
\end{align*}

Corresponding system,
\vspace{-\baselineskip}
\begin{alignat*}{6}
   \alpha_1 &\ +\ & 5 \alpha_2 &\ +\ & 3 \alpha_3 &\ =\ & 0 \\
   && 2\alpha_2 &\ +\ & \alpha_3 &\ =\ & 0 \\
\end{alignat*}

\vspace{-1.5\baselineskip}
The system has a non-trivial solution. Therefore the vectors \ $v_1, \ v_2, \ v_3$ \ are Linearly Dependent.

And, at least one of them is a linear combination of the others.

\begin{center}
   Let, \ $\alpha_3 = -2$ \tabs Now, \ $\alpha_2 = 1, \tab \alpha_1 = 1$

   Therefore, $v_1 \ + \ v_2 \ - \ 2v_3 \ = \ 0$

   $v_1 \ = \ 2v_3 \ - \ v_2$
\end{center}


\begin{center}
Therefore, \ $v_1$ \ is a linear combination of \ $v_2$ \ and \ $v_3$
\end{center}
\end{document}
