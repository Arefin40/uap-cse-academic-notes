\documentclass[12pt]{article}
\usepackage[margin=1.27cm]{geometry}
\usepackage{setspace}
\usepackage{fontspec}
% \usepackage[T1]{fontenc}
% \usepackage[utf8]{inputenc}
\usepackage{amsmath,txfonts,amssymb,nicefrac,mathtools,pifont} %for math
\usepackage{array,tabularx,multirow,fmtcount} %for tables
\usepackage{tikz, pgfplots} %for diagram
\usepackage{multicol} %for multiple column
\usepackage{enumerate,enumitem,adjustbox} %for ordered list
\usepackage{graphicx,subcaption,wrapfig,tcolorbox} %for figure
\usepackage{xparse} %for commands & environments
\usepackage{lipsum} %miscellaneous
\usepackage{colortbl,xcolor,soul} %for default table & border

% #ANCHOR Font settings
\setmainfont{Oxygen}
\newfontfamily\banglafont[Script=Bengali]{Baloo Da 2}
\newfontfamily{\lstsansserif}{IBM Plex Mono}
\renewcommand{\normalsize}{\fontsize{11.5pt}{13pt}\selectfont}


\setlength{\arrayrulewidth}{0.35 pt}
\definecolor{border}{HTML}{A1A1AA}
\arrayrulecolor{border}


% #ANCHOR Document settings
\linespread{1.45}
\setlength\parindent{0pt}
\setlength\parskip{16pt}
\setlist[enumerate]{noitemsep}
\usetikzlibrary{shapes.geometric,decorations.pathreplacing,trees,arrows,positioning,shapes,fit,calc,decorations.markings, decorations.text}
\tikzset{every node/.append style={font=\footnotesize}}
\usepgfplotslibrary{fillbetween}
\pgfdeclarelayer{background}
\pgfsetlayers{background,main}
\pgfplotsset{compat=1.18}
\columnseprule=1pt
\everymath{\displaystyle}
% #ANCHOR Hypernation
\tolerance=1
\emergencystretch=\maxdimen
\hyphenpenalty=10000
\hbadness=10000
\newlength{\colWidth}



% #ANCHOR Colors
\definecolor{azure(colorwheel)}{rgb}{0.0, 0.5, 1.0}
\definecolor{carminepink}{rgb}{0.92, 0.3, 0.26}
\definecolor{orange}{rgb}{0.9, 0.55, 0.22}
\definecolor{violet}{rgb}{0.60, 0.45, 1}
% Syantax Highlighting Colors
\definecolor{keyword}{HTML}{D73A4A}
\definecolor{number}{HTML}{015CC5}
\definecolor{comment}{HTML}{6A737D}
\definecolor{string}{HTML}{1D825E}
\definecolor{function}{HTML}{743FD1}
\definecolor{orange}{HTML}{CF7842}
\definecolor{codeblack}{HTML}{24292F}
\definecolor{divider}{HTML}{A1A1AA}
\definecolor{border}{HTML}{D1D1D1}


% #ANCHOR Ordered & Unordered List
\setlist[itemize,1]{left=0cm, label={\textbullet}}
\setlist[itemize,2,3,4,5,6,7,8,9,10]{left=0.6cm, label={\textbullet}}
\setlist[enumerate,1]{left=0cm}
\setlist[enumerate,2,3,4,5,6,7,8,9,10]{left=0.6cm}
\setul{0.5ex}{0.125ex}



% #ANCHOR Colored Box
\let\oldul\ul
\renewcommand{\ul}[2][keyword]{\text{\setulcolor{#1}\oldul{#2}}}
\newcommand{\redbox}[1]{%
{\color{red}\fbox{\color{black}#1}}
}
\newcommand{\red}[1]{%
\textcolor{red}{#1}
}
\newcommand{\redeq}[1]{%
\text{\color{red}$#1$}
}
\newcommand{\mred}[1]{%
\textcolor{keyword}{#1}
}
\newcommand{\mredeq}[1]{%
\textcolor{keyword}{$#1$}
}
\newcommand{\blue}[1]{%
% {\color{number}#1\hspace{-0.4ex}}
\textcolor{number}{#1}
}
\newcommand{\blueeq}[1]{%
\text{\color{number}$#1$}
}
\newcommand{\cyanbox}[1]{%
{\color{teal}\fbox{\textcolor{black}{#1}}}
}
\newcommand{\cyan}[1]{%
\textcolor{teal}{#1}
}
\newcommand{\pink}[1]{%
\textcolor{magenta}{#1}
}
\newcommand{\orange}[1]{%
\textcolor{orange}{#1}
}
\newcommand{\violet}[1]{%
{\color{violet}#1}
}
\newcommand{\cyaneq}[1]{%
\text{\color{teal}$#1$}
}
\newcommand{\gray}[1]{%
\textcolor{comment}{#1}
}
\newcommand{\pinkeq}[1]{%
\text{\color{magenta}$#1$}
}
\renewcommand{\columnseprulecolor}{\color{divider}}




% #ANCHOR Tabular commands
\newcolumntype{P}[1]{>{\centering\arraybackslash}p{#1}}
\newcolumntype{M}[1]{>{\centering\arraybackslash}m{#1}}
\newcolumntype{C}{>{\centering\arraybackslash}X}
\newcommand{\rspan}[2]{\multirow{#1}{*}{#2}}
\newcommand{\thc}[1]{%
\multicolumn{1}{|c|}{\textbf{#1}}
}
\newcommand{\thcx}[1]{%
\multicolumn{1}{|C|}{\textbf{#1}}
}
\newcommand{\thl}[1]{%
\multicolumn{1}{|l|}{\textbf{#1}}
}
\newcommand{\thr}[1]{%
\multicolumn{1}{|r|}{\textbf{#1}}
}
% Adjusting arraystretch to modify vertical padding
\renewcommand{\arraystretch}{1.25}
% Adjusting tabcolsep to modify horizontal padding
\setlength{\tabcolsep}{10pt}



% #ANCHOR Math commands
\newcommand{\set}[1]{\{$#1$\}}
\newcommand{\tabs}{\ \ \ \ \ \ }
\newcommand{\tab}{\ \ \ }
\newcommand{\cmark}{\ding{51}}%
\newcommand{\xmark}{\ding{55}}%
\newcommand{\boldi}[1]{\boldsymbol{#1}}%
\newcommand{\wspace}{\ \ = \ \ }



% #ANCHOR New commands
\newcommand{\Title}[1]{%
   \begin{center}
      \textbf{\Large{#1}}
   \end{center}
}
\newcommand{\Heading}[1]{%
   \par\vspace{\dimexpr -\baselineskip + 16pt}
   {\fontsize{12pt}{13pt}\selectfont\textbf{#1}}
   \par\vspace{\dimexpr -\baselineskip + 6pt}
}
\newcommand{\BuleHeading}[1]{%
   \par\vspace{\dimexpr -\baselineskip + 16pt}
   {\fontsize{12pt}{13pt}\selectfont\textbf{\textcolor{number}{#1}}}
   \par\vspace{\dimexpr -\baselineskip + 6pt}
}
\newcommand{\CHeading}[1]{%
   \par\vspace{\dimexpr -\baselineskip + 16pt}
   \hspace{\fill}
   {\fontsize{12pt}{13pt}\selectfont\textbf{#1}}
   \hspace{\fill}
   \par\vspace{\dimexpr -\baselineskip + 6pt}
}
\newcommand{\Section}[1]{%
   \par\vspace{\dimexpr -\baselineskip + 16pt}
   \hspace{\fill}
   {\fontsize{13pt}{13pt}\selectfont\textbf{#1}}
   \hspace{\fill}
   \par\vspace{\dimexpr -\baselineskip + 6pt}
}
\newcommand{\seteqno}[1]{%
   \ \cdots \ \cdots \ \cdots \ (#1)
}
\newcommand{\eqor}{%
   \Rightarrow \ \ 
}
\newcommand{\tsub}[1]{%
\textsubscript{#1}\hspace{-0.45ex}
}
\newcommand{\tsup}[1]{%
\textsuperscript{#1}\hspace{-0.45ex}
}
\newcommand{\cbox}[2][cyan]{
\tikz\node[draw=#1,circle,inner sep=2pt,baseline=(a.base)](a){#2};
}
\newcommand{\hrline}{%
\vspace{1ex} {\color{gray}\hrule} \vspace{4ex}
}
\newcommand{\divideX}[1][divider]{{\hspace{1ex}\color{#1}{\vrule}\hspace{1ex}}}
\newcommand{\Reference}[2][Reference]{

\vspace{-0.5\baselineskip}
\begin{center}
   {\fontspec{Merriweather}\textbf{#1:} \textit{#2}} 
\end{center}
}
\newcommand{\bn}[1]{%
   {\banglafont #1}
}

\NewDocumentCommand{\Column}{O{0.49} O{1.5em} m m}{
   \setlength{\colWidth}{\linewidth-#1\linewidth-#2}
   \begin{minipage}[t]{#1\linewidth}
      \noindent
         #3
      \end{minipage}\hspace{\fill}{\color{divider}\vrule width 0.35pt}\hspace{\fill}
      \begin{minipage}[t]{\colWidth}
      \noindent
         #4
   \end{minipage}
}

\usepackage{circuitikz}
\usepackage{tikz}
\newcommand{\ac}[1]{#1 to[sV, fill=white] #1}
\ctikzset{resistors/scale=0.65}
\ctikzset{diodes/scale=0.75}

\begin{document}
% #ANCHOR RTL
\Title{Resistor-Transistor-Logic (RTL) Gates}
\begin{figure}[ht!]
\centering
   \begin{subfigure}[t]{0.3\textwidth}
   \begin{center}
      \begin{circuitikz}
         \draw (0,0) node[npn] (npn) {};
         \draw (npn.base) to[R, anchor=east] ++(-1.5,0) node[left] {$A$};
         \draw (npn.emitter) -- ++(0,-0.5) node[ground] {};
         \draw (npn.collector) to[short,-o] ++(1,0)
         node[above left]{$\bar{A}$};
         \draw (npn.collector) -- ++(0,0.5) to[R] ++(0,1) node[vcc] {$V_{cc}$};
   \end{circuitikz}
   \caption{RTL NOT Gate}
   \end{center}
   \end{subfigure}

   \vspace{10ex}
   \begin{subfigure}[t]{0.3\textwidth}
   \begin{center}
      \begin{circuitikz}
         \draw (0,0) node[npn] (npn1) {};
         \draw (0,-1.5) node[npn] (npn2) {};
         \draw (npn1.base) to[R, anchor=east] ++(-1.5,0) node[left] {$A$};
         \draw (npn1.collector) to[short,-o] ++(1.5,0)
         node[above left]{$\overline{A \cdot B}$};
         \draw (npn1.collector) -- ++(0,0.5) to[R] ++(0,1) node[vcc] {$V_{cc}$};
         \draw (npn2.base) to[R, anchor=east] ++(-1.5,0) node[left] {$B$};
         \draw (npn2.emitter) -- ++(0,-0.5) node[ground] {};
   \end{circuitikz}
   \caption{RTL NAND Gate}
   \end{center}
   \end{subfigure}
   \hspace{0.04\textwidth}
   \begin{subfigure}[t]{0.4\textwidth}
   \begin{center}
      \begin{circuitikz}
         \draw (0,0) node[npn] (npn1) {};
         \draw (2,0) node[npn] (npn2) {};
         \draw (npn1.base) to[R, anchor=east] ++(-1.5,0) node[left] {$A$};
         \draw (npn2.base) -- ++(-0.5,0) -- ++(0,-1) -- ++(-1.5,0) to[R, anchor=east] ++(-1.5,0) node[left] {$B$};
         \draw (npn1.emitter) -- ++(0,-1) -- ++(1,0) node[ground]{} -- ++(1,0) -- (npn2.emitter) {};
         \draw (npn1.collector) -- ++(0,1) -- ++(1,0) coordinate (vcc) {} -- ++(1,0) -- (npn2.collector) {};
         \draw (vcc) to[R] ++(0,2) node[vcc](vcc) {$V_{cc}$};
         \draw (npn2.collector) to[short,-o] ++(1.5,0)
         node[above left] {$\overline{A + B}$};
   \end{circuitikz}
   \caption{RTL NOR Gate}
   \end{center}
   \end{subfigure}
\end{figure}

\pagebreak
% #ANCHOR DTL
\Title{Diode-Transistor-Logic (DTL) Gates}
\begin{figure}[ht!]
\centering
\begin{subfigure}[t]{\textwidth}
\begin{center}
\begin{circuitikz}
   \coordinate (Origin) at (0,-0.5);
   \draw (0,0) node[above](vcc){$V_{cc}$}
   to[short, o-] (Origin)
   (Origin) -| (-1, -1)
   (-1, -1) to[R] (-1,-2.75)
   
   (Origin) -| (1, -1)
   (1, -1) to[R] (1,-2.75)

   (1,-2.75) -- (1, -3)
   node[npn, anchor=collector] (npn){}

   (npn.base) -| node(mid){} (-1,-2.75)
   (npn.base) -- ++ (-1.5,0)
   to[diode, anchor=east] ++(-2,0)
   node[label=left:$A$]{}

   (npn.collector) to[short, -o] ++(1.5,0)
   node[above left]{$\overline{A}$}

   (npn.emitter) node[ground]{}
   (npn.base) ++(-0.5,0) to[diode,fill=white] ++(0.5,0)
   ;
\end{circuitikz}
\caption{DTL NOT Gate}
\end{center}
\end{subfigure}

\vspace{10ex}
\begin{subfigure}[t]{0.4\textwidth}
\begin{center}
\begin{circuitikz}
   \coordinate (Origin) at (0,-0.5);
   \draw (0,0) node[above](vcc){$V_{cc}$}
   to[short, o-] (Origin)
   (Origin) -| (-1, -1)
   (-1, -1) to[R] (-1,-2.75)
   
   (Origin) -| (1, -1)
   (1, -1) to[R] (1,-2.75)

   (1,-2.75) -- (1, -4)
   node[npn, anchor=collector] (npn){}
   
   (-1,-3.5) -- ++(-0.35,0)
   to[diode, anchor=east] ++(-2,0)
   node[label=left:$A$]{}

   (npn.base) -| node(mid){} (-1,-2.75)
   (npn.base) -- ++ (-1.5,0)
   to[diode, anchor=east] ++(-2,0)
   node[label=left:$B$]{}

   (npn.collector) ++(0,0.75) to[short, -o] ++(2,0)
   node[above left]{$\overline{A \cdot B}$}

   (npn.emitter) node[ground]{}
   (npn.base) ++(-0.5,0) to[diode,fill=white] ++(0.5,0)
   ;
\end{circuitikz}
\caption{DTL NAND Gate}
\end{center}
\end{subfigure}
\hspace{5ex}
\begin{subfigure}[t]{0.4\textwidth}
   \begin{center}
   \begin{circuitikz}
      \coordinate (Origin) at (2,0);
      \draw
      (0,0.75) node[label=left:$A$]{}
      to[diode] (2,0.75) -- (Origin)
      
      (0,-0.75) node[label=left:$B$]{}
       to[diode] (2,-0.75) -- (Origin)
   
       (Origin) to[R,anchor=west] (4,0)
       node[npn, anchor=base] (npn){}
   
       (npn.collector) to[R] ++(0,2) to[short, -o] ++(0,0)
       node[above](vcc){$V_{cc}$}
   
       (npn.collector) to[short, -o] ++(2,0)
      node[above left]{$\overline{A + B}$}
   
       (npn.emitter) node[ground]{}
      ;
   \end{circuitikz}
   \caption{DTL NOR Gate}
   \end{center}
   \end{subfigure}
\end{figure}

\pagebreak
% #ANCHOR TTL
\Title{Transistor-Transistor-Logic (TTL) Gates}

\begin{figure}[ht!]
\centering
\begin{subfigure}[t]{\textwidth}
 \begin{center}
   \begin{circuitikz}
      \coordinate (Origin) at (0,-1.5);
      \draw (0,0) node[above](vcc){$V_{cc}$}
      to[R,o-] (Origin)
      (Origin) -| (0.75,-3)
      node[npn, anchor=collector] (npn2){}
      (npn2.base) node[npn, anchor=collector, rotate=-90] (npn1){}
      (npn1.base) |- (Origin)
      (npn1.emitter) to[short] ++(-0.25,0)  node[label=left:$A$]{}
      (npn2.emitter) node[ground]{}
      (npn2.collector) to[short,-o] ++(1,0)
      node[above left]{$\bar{A}$}
      ;
\end{circuitikz}
\end{center}
\caption{TTL NOT Gate}
\end{subfigure}

\vspace{10ex}
\begin{subfigure}[t]{0.4\textwidth}
 \begin{center}
   \begin{circuitikz}
      \coordinate (Origin) at (0,-1.5);
      \draw (0,0) node[above](vcc){$V_{cc}$}
      to[R,o-] (Origin)
      (Origin) -| (0.75,-2.5)
      node[npn, anchor=collector] (npnA2){}

      (npnA2.base) node[npn, anchor=collector, rotate=-90] (npnA1){}
      (npnA1.base) |- (Origin)
      (npnA1.emitter) to[short] ++(-0.25,0)  node[label=left:$A$]{}

      (npnA2.emitter) node[npn, anchor=collector] (npnB2){}
      (npnA2.collector) -- ++(1,0)
      (npnB2.base) node[npn, anchor=collector, rotate=-90] (npnB1){}
      (npnB1.emitter) to[short] ++(-0.25,0)  node[label=left:$B$]{}

      (npnA1.base) -- (npnB1.base)
      (npnB2.emitter) node[ground]{}
      (npnA2.collector) to[short, -o] ++(1.5,0)
      node[above left]{$\overline{A \cdot B}$}
      ;
\end{circuitikz}
\end{center}
\caption{TTL NAND Gate}
\end{subfigure}
\begin{subfigure}[t]{0.45\textwidth}
 \begin{center}
   \begin{circuitikz}
      \coordinate (Origin) at (0,-1.5);
      \coordinate (Center) at (0,-3);
      \draw (0,0) node[above](vcc){$V_{cc}$}
      to[R,o-] (Origin) -- (Center)

      (Center) -| (-0.5,-3.5)
      node[npn, anchor=collector] (npnC1){}
      (Center) -| (0.5,-3.5)
      node[npn, anchor=collector, xscale=-1] (npnC2){}

      (npnC1.base) node[npn, anchor=collector, rotate=-90] (npnA){}
      (npnA.base) |- (Origin)
      (npnA.emitter) to[short] ++(-0.25,0)  node[label=left:$A$]{}

      (npnC2.base) node[npn, anchor=collector, rotate=-90, yscale=-1] (npnB){}
      (npnB.base) |- (Origin)
      (npnB.emitter) to[short] ++(0.25,0)  node[label=right:$B$]{}

      (npnC1.emitter) -- (npnC2.emitter)
      (0,-5.05) node[ground]{}
      (0,-2.5) to[short, -o] ++(4,0)
      node[above left]{$\overline{A + B}$}
      ;
\end{circuitikz}
\end{center}
\caption{TTL NOR Gate}
\end{subfigure}
\end{figure}

\pagebreak
% #ANCHOR NMOS
\Title{NMOS Gates}
\begin{figure}[ht!]
\centering
   \begin{subfigure}[t]{0.3\textwidth}
   \begin{center}
      \begin{circuitikz}
         \draw (0,0) node[nmos] (nmos) {};
         \draw (nmos.base) -- ++(-0.5,0) node[left] {$A$};
         \draw (nmos.emitter) -- ++(0,-0.5) node[ground] {};
         \draw (nmos.collector) to[short,-o] ++(1,0)
         node[above left]{$\bar{A}$};
         \draw (nmos.collector) -- ++(0,0.5) to[R] ++(0,1) node[vcc] {$V_{cc}$};
   \end{circuitikz}
   \caption{NMOS NOT Gate}
   \end{center}
   \end{subfigure}

   \vspace{10ex}
   \begin{subfigure}[t]{0.3\textwidth}
   \begin{center}
      \begin{circuitikz}
         \draw (0,0) node[nmos] (nmos1) {};
         \draw (0,-1.5) node[nmos] (nmos2) {};
         \draw (nmos1.base) -- ++(-0.5,0) node[left] {$A$};
         \draw (nmos1.collector) to[short,-o] ++(1.5,0)
         node[above left]{$\overline{A \cdot B}$};
         \draw (nmos1.collector) -- ++(0,0.5) to[R] ++(0,1) node[vcc] {$V_{cc}$};
         \draw (nmos2.base) -- ++(-0.5,0) node[left] {$B$};
         \draw (nmos2.emitter) -- ++(0,-0.5) node[ground] {};
   \end{circuitikz}
   \caption{NMOS NAND Gate}
   \end{center}
   \end{subfigure}
   \hspace{0.04\textwidth}
   \begin{subfigure}[t]{0.4\textwidth}
   \begin{center}
      \begin{circuitikz}
         \draw (0,0) node[nmos] (nmos1) {};
         \draw (2,0) node[nmos] (nmos2) {};
         \draw (nmos1.base) -- ++(-0.5,0) node[left] {$A$};
         \draw (nmos2.base) -- ++(-0.5,0) -- ++(0,-1) -- ++(-1.5,0) -- ++(-0.5,0) node[left] {$B$};
         \draw (nmos1.emitter) -- ++(0,-1) -- ++(1,0) node[ground]{} -- ++(1,0) -- (nmos2.emitter) {};
         \draw (nmos1.collector) -- ++(0,1) -- ++(1,0) coordinate (vcc) {} -- ++(1,0) -- (nmos2.collector) {};
         \draw (vcc) to[R] ++(0,2) node[vcc](vcc) {$V_{cc}$};
         \draw (nmos2.collector) to[short,-o] ++(1.5,0)
         node[above left] {$\overline{A + B}$};
   \end{circuitikz}
   \caption{NMOS NOR Gate}
   \end{center}
   \end{subfigure}
\end{figure}

\vspace{3ex}
\Heading{Why NMOS technology is prefered over the PMOS?}
The electrons, the majority charge carriers in the NMOS, have a much higher mobility than those of the holes, which form the majority charge carriers in PMOS technology.
\end{document}
