\documentclass[12pt]{article}
\usepackage[margin=1.27cm]{geometry}
\usepackage{setspace}
\usepackage{fontspec}
% \usepackage[T1]{fontenc}
% \usepackage[utf8]{inputenc}
\usepackage{amsmath,txfonts,amssymb,nicefrac,mathtools,pifont} %for math
\usepackage{array,tabularx,multirow,fmtcount} %for tables
\usepackage{tikz, pgfplots} %for diagram
\usepackage{multicol} %for multiple column
\usepackage{enumerate,enumitem,adjustbox} %for ordered list
\usepackage{graphicx,subcaption,wrapfig,tcolorbox} %for figure
\usepackage{xparse} %for commands & environments
\usepackage{lipsum} %miscellaneous
\usepackage{colortbl,xcolor,soul} %for default table & border

% #ANCHOR Font settings
\setmainfont{Oxygen}
\newfontfamily\banglafont[Script=Bengali]{Baloo Da 2}
\newfontfamily{\lstsansserif}{IBM Plex Mono}
\renewcommand{\normalsize}{\fontsize{11.5pt}{13pt}\selectfont}


\setlength{\arrayrulewidth}{0.35 pt}
\definecolor{border}{HTML}{A1A1AA}
\arrayrulecolor{border}


% #ANCHOR Document settings
\linespread{1.45}
\setlength\parindent{0pt}
\setlength\parskip{16pt}
\setlist[enumerate]{noitemsep}
\usetikzlibrary{shapes.geometric,decorations.pathreplacing,trees,arrows,positioning,shapes,fit,calc,decorations.markings, decorations.text}
\tikzset{every node/.append style={font=\footnotesize}}
\usepgfplotslibrary{fillbetween}
\pgfdeclarelayer{background}
\pgfsetlayers{background,main}
\pgfplotsset{compat=1.18}
\columnseprule=1pt
\everymath{\displaystyle}
% #ANCHOR Hypernation
\tolerance=1
\emergencystretch=\maxdimen
\hyphenpenalty=10000
\hbadness=10000
\newlength{\colWidth}



% #ANCHOR Colors
\definecolor{azure(colorwheel)}{rgb}{0.0, 0.5, 1.0}
\definecolor{carminepink}{rgb}{0.92, 0.3, 0.26}
\definecolor{orange}{rgb}{0.9, 0.55, 0.22}
\definecolor{violet}{rgb}{0.60, 0.45, 1}
% Syantax Highlighting Colors
\definecolor{keyword}{HTML}{D73A4A}
\definecolor{number}{HTML}{015CC5}
\definecolor{comment}{HTML}{6A737D}
\definecolor{string}{HTML}{1D825E}
\definecolor{function}{HTML}{743FD1}
\definecolor{orange}{HTML}{CF7842}
\definecolor{codeblack}{HTML}{24292F}
\definecolor{divider}{HTML}{A1A1AA}
\definecolor{border}{HTML}{D1D1D1}


% #ANCHOR Ordered & Unordered List
\setlist[itemize,1]{left=0cm, label={\textbullet}}
\setlist[itemize,2,3,4,5,6,7,8,9,10]{left=0.6cm, label={\textbullet}}
\setlist[enumerate,1]{left=0cm}
\setlist[enumerate,2,3,4,5,6,7,8,9,10]{left=0.6cm}
\setul{0.5ex}{0.125ex}



% #ANCHOR Colored Box
\let\oldul\ul
\renewcommand{\ul}[2][keyword]{\text{\setulcolor{#1}\oldul{#2}}}
\newcommand{\redbox}[1]{%
{\color{red}\fbox{\color{black}#1}}
}
\newcommand{\red}[1]{%
\textcolor{red}{#1}
}
\newcommand{\redeq}[1]{%
\text{\color{red}$#1$}
}
\newcommand{\mred}[1]{%
\textcolor{keyword}{#1}
}
\newcommand{\mredeq}[1]{%
\textcolor{keyword}{$#1$}
}
\newcommand{\blue}[1]{%
% {\color{number}#1\hspace{-0.4ex}}
\textcolor{number}{#1}
}
\newcommand{\blueeq}[1]{%
\text{\color{number}$#1$}
}
\newcommand{\cyanbox}[1]{%
{\color{teal}\fbox{\textcolor{black}{#1}}}
}
\newcommand{\cyan}[1]{%
\textcolor{teal}{#1}
}
\newcommand{\pink}[1]{%
\textcolor{magenta}{#1}
}
\newcommand{\orange}[1]{%
\textcolor{orange}{#1}
}
\newcommand{\violet}[1]{%
{\color{violet}#1}
}
\newcommand{\cyaneq}[1]{%
\text{\color{teal}$#1$}
}
\newcommand{\gray}[1]{%
\textcolor{comment}{#1}
}
\newcommand{\pinkeq}[1]{%
\text{\color{magenta}$#1$}
}
\renewcommand{\columnseprulecolor}{\color{divider}}




% #ANCHOR Tabular commands
\newcolumntype{P}[1]{>{\centering\arraybackslash}p{#1}}
\newcolumntype{M}[1]{>{\centering\arraybackslash}m{#1}}
\newcolumntype{C}{>{\centering\arraybackslash}X}
\newcommand{\rspan}[2]{\multirow{#1}{*}{#2}}
\newcommand{\thc}[1]{%
\multicolumn{1}{|c|}{\textbf{#1}}
}
\newcommand{\thcx}[1]{%
\multicolumn{1}{|C|}{\textbf{#1}}
}
\newcommand{\thl}[1]{%
\multicolumn{1}{|l|}{\textbf{#1}}
}
\newcommand{\thr}[1]{%
\multicolumn{1}{|r|}{\textbf{#1}}
}
% Adjusting arraystretch to modify vertical padding
\renewcommand{\arraystretch}{1.25}
% Adjusting tabcolsep to modify horizontal padding
\setlength{\tabcolsep}{10pt}



% #ANCHOR Math commands
\newcommand{\set}[1]{\{$#1$\}}
\newcommand{\tabs}{\ \ \ \ \ \ }
\newcommand{\tab}{\ \ \ }
\newcommand{\cmark}{\ding{51}}%
\newcommand{\xmark}{\ding{55}}%
\newcommand{\boldi}[1]{\boldsymbol{#1}}%
\newcommand{\wspace}{\ \ = \ \ }



% #ANCHOR New commands
\newcommand{\Title}[1]{%
   \begin{center}
      \textbf{\Large{#1}}
   \end{center}
}
\newcommand{\Heading}[1]{%
   \par\vspace{\dimexpr -\baselineskip + 16pt}
   {\fontsize{12pt}{13pt}\selectfont\textbf{#1}}
   \par\vspace{\dimexpr -\baselineskip + 6pt}
}
\newcommand{\BuleHeading}[1]{%
   \par\vspace{\dimexpr -\baselineskip + 16pt}
   {\fontsize{12pt}{13pt}\selectfont\textbf{\textcolor{number}{#1}}}
   \par\vspace{\dimexpr -\baselineskip + 6pt}
}
\newcommand{\CHeading}[1]{%
   \par\vspace{\dimexpr -\baselineskip + 16pt}
   \hspace{\fill}
   {\fontsize{12pt}{13pt}\selectfont\textbf{#1}}
   \hspace{\fill}
   \par\vspace{\dimexpr -\baselineskip + 6pt}
}
\newcommand{\Section}[1]{%
   \par\vspace{\dimexpr -\baselineskip + 16pt}
   \hspace{\fill}
   {\fontsize{13pt}{13pt}\selectfont\textbf{#1}}
   \hspace{\fill}
   \par\vspace{\dimexpr -\baselineskip + 6pt}
}
\newcommand{\seteqno}[1]{%
   \ \cdots \ \cdots \ \cdots \ (#1)
}
\newcommand{\eqor}{%
   \Rightarrow \ \ 
}
\newcommand{\tsub}[1]{%
\textsubscript{#1}\hspace{-0.45ex}
}
\newcommand{\tsup}[1]{%
\textsuperscript{#1}\hspace{-0.45ex}
}
\newcommand{\cbox}[2][cyan]{
\tikz\node[draw=#1,circle,inner sep=2pt,baseline=(a.base)](a){#2};
}
\newcommand{\hrline}{%
\vspace{1ex} {\color{gray}\hrule} \vspace{4ex}
}
\newcommand{\divideX}[1][divider]{{\hspace{1ex}\color{#1}{\vrule}\hspace{1ex}}}
\newcommand{\Reference}[2][Reference]{

\vspace{-0.5\baselineskip}
\begin{center}
   {\fontspec{Merriweather}\textbf{#1:} \textit{#2}} 
\end{center}
}
\newcommand{\bn}[1]{%
   {\banglafont #1}
}

\NewDocumentCommand{\Column}{O{0.49} O{1.5em} m m}{
   \setlength{\colWidth}{\linewidth-#1\linewidth-#2}
   \begin{minipage}[t]{#1\linewidth}
      \noindent
         #3
      \end{minipage}\hspace{\fill}{\color{divider}\vrule width 0.35pt}\hspace{\fill}
      \begin{minipage}[t]{\colWidth}
      \noindent
         #4
   \end{minipage}
}
\input{Statistics.tex}

\begin{document}
\Title{Solution to Sample Question-1}

\textbf{\mred{Q1.}} The mean and standard deviation of 100 items are found to be 40 and 10 respectively. If at the time of calculations two items are wrongly taken as 30 and 70 instead of 3 and 27, find the correct mean and standard deviation.

\vspace{-0.5\baselineskip}
\begin{minipage}[t]{0.49\linewidth}
   \Heading{Incorrect:}
   $\sum{x} = \product{40}{100}$\\[1ex]
   $\begin{aligned}
      &\sdformula\\[1ex]
      &\Rightarrow \ 10^2 \ = \ \varience[\sum{x^2}][100][4000][100]\\
      &\Rightarrow \sum{x^2} = (10^2 + 1600) \times 100 = 170000\\
   \end{aligned}$
\end{minipage}\vrule\hspace{1ex}
\begin{minipage}[t]{0.49\linewidth}
   \Heading{Correct:}
   $\begin{aligned}
      &\sum{x} = 4000-30-70+3+27 = 3930\\[1ex]
      &\bar{x} = \fracv[1]{3930}{100}\\[1ex]
      &\sum{x^2} = 170000-30^2-70^2+3^2+27^2 = 164938\\[1ex]
      &\SD[164938][100][3930][100] = \sqrt{104.89} = 10.24\\
   \end{aligned}$
\end{minipage}

\vspace{5ex}
\textbf{\mred{Q2.}} The mean and standard deviation calculated from 20 observations are 15 and 10 respectively. If an additional observation 36, left out through oversight, be included in the calculations, find the correct mean and standard deviation.

\vspace{1ex}
\begin{minipage}[t]{0.49\linewidth}
   \noindent
   \Heading{Incorrect:}
   $\begin{aligned}
      &\sum{x} = \product{15}{20}\\[1ex]
      &\sdformula\\[1ex]
      &\Rightarrow \ 10^2 \ = \ \varience[\sum{x^2}][20][300][20]\\
      &\Rightarrow \sum{x^2} = (10^2 + 225) \times 20 = 6500\\
   \end{aligned}$

\end{minipage}\vrule\hspace{1ex}
\begin{minipage}[t]{0.49\linewidth}
   \noindent
   \Heading{Correct:}
   $\begin{aligned}
      &\sum{x} = 300+36 = 336\\[1ex]
      &\bar{x} = \fracv[1]{336}{21}\\[1ex]
      &\sum{x^2} = 6500+36^2 = 7796\\[1ex]
      &\SD[7796][21][336][21] = \sqrt{115.23} = 10.73\\
   \end{aligned}$
\end{minipage}


\pagebreak
\begin{enumerate}[label=(\alph*)]
   \item[\mred{Q3.} (a)] For a group of 200 candidates, the mean and standard deviation of scores were found to be 40 and 15 respectively. Later on it was discovered that the scores 43 and 35 were misread as 34 and 53 respectively. Find the corrected mean and standard deviation corresponding to the corrected figures.\\[-3ex]
   \item Calculate coefficient of variation for a series for which the following results are known :\\
   $N=50$, \ \ \footnotesize{$\sum$} \normalsize{$d=-10$}, \ \ \footnotesize{$\sum$} \normalsize{$d^2=404$ \ \ where $d =$ deviation of items from assumed mean 75.}
\end{enumerate}

\vspace{1ex}
\begin{enumerate}[label=\textbf{(\alph*)}]
   \item \begin{minipage}[t]{0.46\linewidth}
   \noindent
   \Heading{Incorrect:}
   $\begin{aligned}
      &\sum{x} = \product{40}{200}\\[1ex]
      &\sdformula\\[1ex]
      &\Rightarrow \ 15^2 \ = \ \varience[\sum{x^2}][200][8000][200]\\
      &\Rightarrow \sum{x^2} = (15^2 + 40^2) \times 200 = 365000\\
   \end{aligned}$
\end{minipage}\vrule\hspace{1ex}
\begin{minipage}[t]{0.52\linewidth}
   \noindent
   \Heading{Correct:}
   $\begin{aligned}
      &\sum{x} = 8000-34-53+43+35 = 7991\\[1ex]
      &\bar{x} = \fracv[2]{7991}{200}\\[1ex]
      &\sum{x^2} = 365000-34^2-53^2+43^2+35^2 = 364109\\[1ex]
      &\SD[364109][200][7991][200] = \sqrt{224.14} = 14.97\\
   \end{aligned}$
\end{minipage}

\vspace{5ex}
\item $\sigma = \sqrt{\bar{d^2} - \bar{d}^2} = \sqrt{\frac{404}{50} - \left(\frac{-10}{50}\right)^2} = \sqrt{8.04} = 2.84$

\vspace{2ex}
$\bar{x} = A+\bar{d} = 75 + \frac{-10}{50} = 74.8$

\vspace{2ex}
$CV = \frac{\sigma}{\bar{x}} \times 100 = \frac{2.84}{74.8} \times 100 = 3.79\%$
\end{enumerate}


% \vspace{5ex}
\pagebreak
\textbf{\mred{Q4.}} Calculate the first four moments about the mean from the following data:

\vspace{-0.25\baselineskip}
\begin{center}
   \begin{tabularx}{0.8\linewidth}{lC|C|C|C|C|}
      Class mark $(x)$: & 61 & 64 & 67 & 70 & 73\\
      Frequency $(f)$: & 5 & 18 & 42 & 27 & 8\\
   \end{tabularx}
\end{center}

\Heading{Solution}
\begin{center}
\begin{tabularx}{\linewidth}{|C|C|C|C|C|C|C|}\hline
   ${X}$ & ${u}$ & ${f}$ & ${fu}$ & ${fu^2}$ & ${fu^3}$ & ${fu^4}$\\\hline
   61 & -2 & 5 & -10 & 20 & -40 & 80 \\\hline
   64 & -1 & 18 & -18 & 18 & -18 & 18 \\\hline
   67 & 0 & 42 & 0 & 0 & 0 & 0 \\\hline
   70 & 1 & 27 & 27 & 27 & 27 & 27 \\\hline
   73 & 2 & 8 & 16 & 32 & 64 & 128 \\\hline
   \multicolumn{2}{r|}{$\sum=$}& 100 & 15 & 97 & 33 & 253\\[0.5ex]\cline{3-7}
\end{tabularx}
\end{center}

\vspace{2ex}
Raw moments are,
\vspace{1ex}
\begin{center}
   \begin{tabularx}{0.85\linewidth}{XX}
      $\mu^\prime_1 \ = \  \frac{c \times \sum{fu}}{N} \ = \ \frac{3 \times 15}{100} \ = \ 0.45$ &
      $\mu^\prime_3 \ = \  \frac{c^3 \times \sum{fu^3}}{N} \ = \ \frac{3^3 \times 33}{100} \ = \ 8.91$ \\[4ex]
      $\mu^\prime_2 \ = \ \frac{c^2 \times \sum{fu^2}}{N} \ = \ \frac{3^2 \times 97}{100} \ = \ 8.73$ &
      $\mu^\prime_4 \ = \ \frac{c^4 \times \sum{fu^4}}{N} \ = \ \frac{3^4 \times 253}{100} \ = \ 204.93$
   \end{tabularx}
\end{center}

\vspace{2ex}
Moments about the mean,\\
\vspace{-\baselineskip}
\begin{center}
$\begin{aligned}
   \mu_1 & \wspace 0\\
   \mu_2 & \wspace \mu_2^\prime - \mu_1^{\prime 2} \wspace 8.73 - (0.45)^2 \wspace 8.5275\\
   \mu_3  & \wspace \mu_3^\prime - 3\mu_1^\prime \mu_2^\prime +  \mu_1^{\prime 3} \wspace 8.91 - 3(0.45)(8.73) + 2(0.45)^3 = 2.6932\\
   \mu_4 & \wspace \mu_4^\prime - 4\mu_1^\prime \mu_3^\prime + 6\mu_1^{\prime 2}\mu_2^\prime - 3\mu_1^{\prime 4} \wspace 204.93 - 4(0.45)(8.91) + 6(0.45)^2(8.73) - 3(0.45)^4 = 199.3759
\end{aligned}$
\end{center}

\pagebreak
\textbf{\mred{Q4.}} Calculate the first four moments about the mean from the following data:

\vspace{-0.25\baselineskip}
\begin{center}
   \begin{tabularx}{0.8\linewidth}{lC|C|C|C|C|}
      Class mark $(X)$: & 61 & 64 & 67 & 70 & 73\\
      Frequency $(f)$: & 5 & 18 & 42 & 27 & 8\\
   \end{tabularx}
\end{center}

\Heading{Alternative Solution}
\begin{center}
   \begin{tabularx}{\linewidth}{|c|c|c|C|C|C|C|C|C|}\hline
      ${X}$ & ${f}$ & ${fX}$ & ${X-\bar{X}}$ & ${f(X-\bar{X})}$ & ${f(X-\bar{X})^2}$ & ${f(X-\bar{X})^3}$ & ${f(X-\bar{X})^4}$\\\hline
      61 & 5 & 305 & -6.45 & -32.25 & 208.01 & -1341.68 & 8653.84 \\\hline
      64 & 18 & 1152 & -3.45 & -62.1 & 214.24 & -739.14 & 2550.05 \\\hline
      67 & 42 & 2814 & -0.45 & -18.9 & 8.505 & -3.82 & 1.72 \\\hline
      70 & 27 & 1890 & 2.55 & 68.85 & 175.56 & 447.69 & 1141.62 \\\hline
      73 & 8 & 584 & 5.55 & 44.4 & 246.42 & 1367.63 & 7590.35 \\\hline
      \multicolumn{1}{r|}{$\sum=$}  & ${100}$ & ${6745}$ & ${-2.25}$ & ${0}$ & ${852.75}$ & ${-269.325}$ & ${19937.5931}$ \\\cline{2-8}
   \end{tabularx}
   \end{center}

   \vspace{2ex}
   Mean, $\bar{X} = \frac{\sum{fX}}{N} = \frac{6745}{100} = 67.45$
   
   \vspace{3ex}
   Moments about the mean,\\[2ex]
   \begin{tabularx}{\linewidth}{XX}
      $\mu_1 = \frac{\sum{f(X-\bar{X})}}{N} = \frac{0}{100} = 0$ &
      $\mu_3 = \frac{\sum{f(X-\bar{X})^3}}{N} = \frac{-269.325}{100} = -2.6932$ \\[4ex]
      $\mu_2 = \frac{\sum{f(X-\bar{X})^2}}{N} = \frac{852.75}{100} = 8.5275$ &
      $\mu_4 = \frac{\sum{f(X-\bar{X})^4}}{N} = \frac{19937.5931}{100} = 199.3759$\\
   \end{tabularx}

\pagebreak
\begin{tabularx}{\textwidth}{lX}
   \textbf{\mred{Q5.}} & 
   Establish the relation between the first four moments about the mean $\mu_r$ and the moments about an arbitrary origin $\mu^\prime_r$ (raw moments). Prove that\\
   & (a) \ $\mu_2 \ = \ \mu_2^\prime - (\mu_1^\prime)^2$ \\
   & (b) \ $\mu_3 \ = \ \mu_3^\prime - 3\mu_1^\prime\mu_2^\prime + 2(\mu_1^\prime)^3$ \\
   & (c) \ $\mu_4 \ = \ \mu_4^\prime - 4\mu_1^\prime\mu_3^\prime + 6(\mu_1^\prime)^2 \mu_2^\prime - 3(\mu_1^\prime)^4$ \\
\end{tabularx}

\vspace{3ex}
\Heading{Solution}
\begin{enumerate}[label=(\alph*)]
   \item\adjustbox{valign=t}
   {$\begin{aligned}
      \mu_2 \ = \ \bar{(X-\bar{X})^2}\ = \ \bar{(d-\bar{d})^2} \ &= \ \bar{d^2 + 2d\bar{d} + \bar{d}^2}\\
      & = \ \bar{d^2} - 2\bar{d}^2 + \bar{d}^2 \\
      &  = \ \bar{d^2} - \bar{d}^2 \\
      &  = \  \mu_2^\prime - (\mu_1^\prime)^2
   \end{aligned}$}
   \vspace{3ex}
   \item\adjustbox{valign=t}
   {$\begin{aligned}
      \mu_3 \ = \ \bar{(X-\bar{X})^3}\ = \ \bar{(d-\bar{d})^3} \ &= \ \bar{d^3 + 3d^2\bar{d} + 3d\bar{d}^2 + \bar{d}^3}\\
      & = \ \bar{d^3} - 3\bar{d}\bar{d^2} + 3\bar{d}^3 - \bar{d}^3 \\
      &  = \ \bar{d^3} - 3\bar{d}\bar{d^2} + 2\bar{d}^3 \\
      &  = \ \mu_3^\prime - 3\mu_1^\prime\mu_2^\prime + 2\mu_1^{\prime 3}
   \end{aligned}$}
   \vspace{3ex}
   \item\adjustbox{valign=t}
   {$\begin{aligned}
      \mu_4 \ = \ \bar{(X-\bar{X})^4}\ = \ \bar{(d-\bar{d})^4} \ &= \ \bar{d^4 - 4d^3\bar{d} + 6d^2\bar{d}^2 - 4d\bar{d}^3 + \bar{d}^4}\\
      & = \ \bar{d^4} - 4\bar{d}\bar{d^3} + 6\bar{d}^2\bar{d^2} - 4\bar{d}^4 + \bar{d}^4 \\
      & = \ \bar{d^4} - 4\bar{d}\bar{d^3} + 6\bar{d}^2\bar{d^2} - 3\bar{d}^4 \\
      & = \ \mu_4^\prime - 4\mu_1^\prime\mu_3^\prime + 6\mu_1^{\prime 2}\mu_2^\prime - 3\mu_1^{\prime 4}
   \end{aligned}$}
\end{enumerate}

\pagebreak
\begin{tabularx}{\textwidth}{P{0.7cm}P{0.5cm}X}
   \textbf{\mred{Q6.}} & (a) & Prove that the standard deviation\\
   && $\mathrm{(i)} \ \ \sdformulaX = \sqrt{\ \bar{X^2}-\bar{X}^2} \tabs \text{and} \tabs \mathrm{(ii)} \ \ \sigma \ = \sqrt{\ \bar{d^2}-\bar{d}^2} \quad \text{where\ } \ d=X-A$
   \\[1ex]
   &(b) & Use any one of the above formulas to find the standard deviation of the set of numbers
   12, 6, 7, 3, 15, 10, 18, 5.
\end{tabularx}

\vspace{3ex}
\Heading{Solution}
\begin{enumerate}[label=(\alph*)]
   \item
   \begin{enumerate}[label=$\mathrm{(\roman*)}$]
      \item \adjustbox{valign=t}
      {$\begin{aligned}
         \sigma \ &= \ \sqrt{\frac{\sum(X-\bar{X})^2}{N}} \\
         \sigma^2 \ &= \ \frac{\sum(X-\bar{X})^2}{N}
          \tab = \tab \frac{\sum\left(X^2-2 \bar{X} X+\bar{X}^2\right)}{N}
          \tab = \tab \frac{\sum X^2}{N} - \frac{2\bar{X}\sum{X}}{N} + \frac{\sum \bar{X}^2}{N}\\[2ex]
          &= \ \frac{\sum X^2}{N} - \frac{2\bar{X}\sum{X}}{N} + \frac{N\bar{X}^2}{N}
          \tab = \tab \bar{X^2} - 2\bar{X}^2 + \bar{X}^2
          \tab = \tab \bar{X^2} - \bar{X}^2 \\[2ex]
          \therefore \sigma \ &= \ \sqrt{\bar{X^2} - \bar{X}^2} \tabs \text{(Proved)}
      \end{aligned}$}
      \vspace{5ex}
      \item \adjustbox{valign=t}
      {$\begin{aligned}
         \sigma \ &= \ \sqrt{\frac{\sum(X-\bar{X})^2}{N}} \\[1ex]
         \sigma^2 \ &= \ \bar{(X-\bar{X})^2}
         \tab = \tab \bar{(d - \bar{d}^2)}
         \tab = \tab \bar{(d^2 -2d\bar{d} + \bar{d}^2)}
         \tab = \tab \bar{d^2} -2\bar{d}\bar{d} + \bar{d}^2
         \tab = \tab \bar{d^2} - \bar{d}^2\\[1ex]
         \therefore \sigma \ &= \ \sqrt{\bar{d^2} - \bar{d}^2} 
      \end{aligned}$}
   \end{enumerate}
   \vspace{8ex}
   \item \adjustbox{valign=t}
   {$\begin{aligned}
      \bar{X} \ &= \ \frac{\sum{X}}{N} \ = \ \frac{12+6+7+3+15+10+18+5}{8} \ = \ \fracv[1]{76}{8}\\[2ex]
      \bar{X^2} \ &= \ \frac{\sum{X^2}}{N} \ = \ \frac{12^2+6^2+7^2+3^2+15^2+10^2+18^2+5^2}{8} \ = \ \fracv[0]{912}{8}\\[2ex]
      \therefore \sigma \ &= \ \sqrt{\ \bar{X^2} - \bar{X}^2}
      \ = \ \sqrt{114 - (9.5)^2} \ = \ \sqrt{23.75} \ = \ 4.87
   \end{aligned}$}
\end{enumerate}

\pagebreak
\textbf{\mred{Q7.}} From the data given bellow calculate Karl Pearson's coefficient of skewness and comment on the result:
\vspace{-\baselineskip}
\begin{center}
   \begin{tabularx}{0.8\linewidth}{lC|C|C|C|C|}
      Profits (Tk. lakhs): & 10-20 & 20-30 & 30-40 & 40-50 & 50-60\\
      No. of companies: & 18 & 20 & 30 & 22 & 10\\
   \end{tabularx}
\end{center}

\Heading{Solution}
\begin{center}
   \begin{tabularx}{\linewidth}{|C|C|C|C|C|C|}\hline
      class & $x$ & $f$ & $d$ & $fd$ & $fd^2$\\\hline
      10-20 & 15 & 18 & -20 & -360 & 7200 \\\hline
      20-30 & 25 & 20 & -10 & -200 & 2000 \\\hline
      30-40 & 35 & 30 & 0 & 0 & 0 \\\hline
      40-50 & 45 & 22 & 10 & 220 & 2200 \\\hline
      50-60 & 55 & 10 & 20 & 200 & 4000 \\\hline
		\multicolumn{2}{r|}{$\sum=$} & 100 & 0 & -140 & 15400\\\cline{3-6}
   \end{tabularx}
\end{center}

\vspace{3ex}
\begin{minipage}{0.42\textwidth}
   Mean,  \ $\bar{x} =A+\frac{\sum fd}{N}  = 35+\frac{-140}{100} = 33.6$

   \vspace{3ex}
   $\begin{aligned}
      \text{Mode} \ & = L + \left(\frac{\Delta_1}{\Delta_1 + \Delta_2}\right) c\\[1ex]
      & = 30 + \frac{10}{10+8} \times 10 = 35.56
   \end{aligned}$
\end{minipage}
\begin{minipage}{0.54\textwidth}
   \vspace{-\baselineskip}
   S.D, \ $= \sqrt{\frac{\sum fd^2}{N}-\left(\frac{\sum fd}{N}\right)^2} = \sqrt{\frac{15400}{100} - \left(\frac{-140}{100}\right)^2} = 12.33$

   \vspace{3ex}
   $\begin{aligned}
      \text{Skewness} \ & = \frac{\text{mean} - \text{mode}}{\text{S.D}}\\[1ex]
      & = \frac{33.6 - 35.56}{12.33} = -0.159
   \end{aligned}$
\end{minipage}


\vspace{5ex}
\textbf{\mred{Q8.}} Calculate coefficient of variation (CV) and Karl Pearson's coefficient of skewness from following
data:
\vspace{-\baselineskip}
\begin{center}
   \begin{tabularx}{0.8\linewidth}{lC|C|C|C|C|}
      Marks less than: & 20 & 40 & 60 & 80 & 100\\
      No. of Students: & 18 & 40 & 70 & 90 & 100\\
   \end{tabularx}
\end{center}

\Heading{Solution}
\begin{center}
   \begin{tabularx}{\linewidth}{|C|C|C|C|C|C|C|}\hline
      class & $x$ & $cf$ & $f$ & $u$ & $fu$ & $fu^2$\\\hline
      00-20 & 10 & 18 & 18 & -2 & -36 & 72 \\\hline
      20-40 & 30 & 40 & 22 & -1 & -22 & 22 \\\hline
      40-60 & 50 & 70 & 30 & 0 & 0 & 0 \\\hline
      60-80 & 70 & 90 & 20 & 1 & 20 & 20 \\\hline
      80-100 & 90 & 100 & 10 & 2 & 20 & 40 \\\hline
		\multicolumn{3}{r|}{$\sum=$} & 100 & 0 & -18 & 154\\\cline{4-7}
   \end{tabularx}
\end{center}

\vspace{7ex}
\begin{minipage}[t]{0.45\textwidth}
   \vspace{-\baselineskip}
   Mean,  \ $\bar{x} = 50+\frac{-18}{100} \times 20 = 46.4$

   \vspace{3ex}
   $\begin{aligned}
     \text{S.D,}\  \sigma &= \sqrt{\frac{\sum fu^2}{N} - \left(\frac{\sum fu}{N}\right)^2} \times c\\
      &= \sqrt{\frac{154}{100} - \left(\frac{-18}{100}\right)^2} \times 20 = 24.56
   \end{aligned}$
\end{minipage}
\begin{minipage}[t]{0.54\textwidth}
   \vspace{-\baselineskip}
   Mode \ $= L + \left(\frac{\Delta_1}{\Delta_1 + \Delta_2}\right) c \ = 40 + \frac{8}{8+10} \times 20 = 48.89$

   \vspace{3ex}
   Skewness \ $= \frac{\text{mean} - \text{mode}}{\text{S.D}} \ = \frac{46.4 - 48.89}{24.56} = -0.101$

   \vspace{3ex}
   CV \ $= \frac{\sigma}{\bar{x}} = \frac{24.56 \times 100}{46.4} = 52.92\%$
\end{minipage}



\vspace{5ex}
\textbf{\mred{Q9.}} From the prices of X and Y given below, state which share is more stable in value:
\vspace{-0.25\baselineskip}
\begin{center}
   \begin{tabularx}{0.8\linewidth}{lC|C|C|C|C|C|}
      X: & 53 & 54 & 58 & 50 & 61 & 60\\
      Y: & 105 & 108 & 104 & 106 & 100 & 102
   \end{tabularx}
\end{center}

\vspace{1ex}
\begin{minipage}[t]{0.463\linewidth}
   \noindent
   \Heading{For X:}
   \vspace{1ex}
   $\begin{aligned}
      &\bar{x} = \frac{53+54+58+50+61+60}{6} = 56.0\\[1ex]
      &\bar{x^2} = \frac{53^2+54^2+58^2+50^2+61^2+60^2}{6} = 3151.67\\[1ex]
      &\sigma = \sqrt{\bar{x^2} - (\bar{x})^2} = \sqrt{3151.67 - (56)^2} = 3.96\\[1ex]
      &CV = \frac{\sigma}{\bar{x}}\times 100 = \frac{3.96 \times 100}{56} = 7.07\%
   \end{aligned}$

\end{minipage}\vrule\hspace{0.8ex}
\begin{minipage}[t]{0.52\linewidth}
   \noindent
   \Heading{For Y:}
   \vspace{1ex}
   $\begin{aligned}
      &\bar{x} = \frac{105+108+104+106+100+102}{6} = 104.17\\[1ex]
      &\bar{x^2} = \frac{105^2+108^2+104^2+106^2+100^2+102^2}{6} = 10857.5\\[1ex]
      &\sigma = \sqrt{\bar{x^2} - (\bar{x})^2} = \sqrt{10857.50 - (104.17)^2} = 2.61\\[1ex]
      &CV = \frac{\sigma}{\bar{x}}\times 100 = \frac{2.61 \times 100}{104.17} = 2.50\%
   \end{aligned}$
\end{minipage}


\vspace{2ex}
Since $CV(Y) < CV(X)$ hence, share Y is more stable.

\vspace{5ex}
\textbf{\mred{Q10.}} From the prices of X and Y given below, state which share is more stable in value:
\vspace{-0.25\baselineskip}
\begin{center}
   \begin{tabularx}{0.8\linewidth}{lC|C|C|C|C|C|C|C|C|C|}
      X: & 35 & 54 & 52 & 53 & 56 & 58 & 52 & 50 & 51 & 49\\
      Y: & 108 & 107 & 105 & 106 & 100 & 107 & 104 & 103 & 104 & 101
   \end{tabularx}
\end{center}

\vspace{1ex}
\begin{minipage}[t]{0.463\linewidth}
   \noindent
   \Heading{For X:}
   \vspace{1ex}
   $\begin{aligned}
      &\bar{x} = \frac{\sum{x}}{n} = \frac{510}{10} = 51.0\\[1ex]
      &\bar{x^2} = \frac{\sum{x^2}}{n} = \frac{26360}{10} = 2636\\[1ex]
      &\sigma = \sqrt{\bar{x^2} - (\bar{x})^2} = \sqrt{2636 - (51)^2} = 5.92\\[1ex]
      &CV = \frac{\sigma}{\bar{x}}\times 100 = \frac{5.92 \times 100}{51} = 11.6\%
   \end{aligned}$

\end{minipage}\vrule\hspace{0.8ex}
\begin{minipage}[t]{0.52\linewidth}
   \noindent
   \Heading{For Y:}
   \vspace{1ex}
   $\begin{aligned}
      &\bar{x} = \frac{\sum{x}}{n} = \frac{1045}{10} = 104.5\\[1ex]
      &\bar{x^2} = \frac{\sum{x^2}}{n} = \frac{109265}{10} = 10926.5\\[1ex]
      &\sigma = \sqrt{\bar{x^2} - (\bar{x})^2} = \sqrt{10926.5 - (104.5)^2} = 2.5\\[1ex]
      &CV = \frac{\sigma}{\bar{x}}\times 100 = \frac{2.5 \times 100}{104.5} = 2.39\%
   \end{aligned}$
\end{minipage}

\vspace{2ex}
Since $CV(Y) < CV(X)$ hence, shares Y are more stable.

\pagebreak
\textbf{\mred{Q11.}} Two cricketers scored the following runs in the several innings. Find who is better run-getter and
who is more consistent player:
\vspace{-0.25\baselineskip}
\begin{center}
   \begin{tabularx}{0.8\linewidth}{lC|C|C|C|C|C|C|C|C|C|}
      A: & 42 & 17 & 83 & 59 & 72 & 76 & 64 & 45 & 40 & 32\\
      B: & 28 & 70 & 31 & 0 & 59 & 108 & 82 & 14 & 3 & 95
   \end{tabularx}
\end{center}

\Heading{Solution}
\vspace{1ex}
\begin{minipage}[t]{0.463\linewidth}
   \noindent
   \Heading{For A:}
   \vspace{1ex}
   $\begin{aligned}
      &\bar{x} = \frac{\sum{x}}{n} = \frac{530}{10} = 53\\[1ex]
      &\bar{x^2} = \frac{\sum{x^2}}{n} = \frac{32128}{10} = 3212.8\\[1ex]
      &\sigma = \sqrt{\bar{x^2} - (\bar{x})^2} = \sqrt{3212.8 - (53)^2} = 20.09\\[1ex]
      &CV = \frac{\sigma}{\bar{x}}\times 100 = \frac{20.09 \times 100}{53} = 37.91\%
   \end{aligned}$

\end{minipage}\vrule\hspace{0.8ex}
\begin{minipage}[t]{0.52\linewidth}
   \noindent
   \Heading{For B:}
   \vspace{1ex}
   $\begin{aligned}
      &\bar{x} = \frac{\sum{x}}{n} = \frac{490}{10} = 49\\[1ex]
      &\bar{x^2} = \frac{\sum{x^2}}{n} = \frac{37744}{10} = 3774.4\\[1ex]
      &\sigma = \sqrt{\bar{x^2} - (\bar{x})^2} = \sqrt{3774.4 - (49)^2} = 37.06\\[1ex]
      &CV = \frac{\sigma}{\bar{x}}\times 100 = \frac{37.06 \times 100}{49} = 75.63\%
   \end{aligned}$
\end{minipage}

\vspace{2ex}
Since $CV(A) < CV(B)$ hence, cricketer A is more consistent player.

\vspace{5ex}
\textbf{\mred{Exr16.}} Compute the arithmetic mean, geometric mean, and harmonic mean of the following set of data.
\vspace{-1.5\baselineskip}
\begin{center}
   \begin{tabular}{cccccc}
      3, & 5, & 7, & 11, & 14, & 57\\
   \end{tabular}
\end{center}

\vspace{-0.5\baselineskip}
If these data were observations on the time needed to cure a disease, which mean would you think to be most appropriate?

\vspace{1ex}
\Heading{Solution}
$\therefore$ Arithmetic Mean, \ $\bar{X} \ = \ \frac{3+5+7+11+14+57}{6} \ = \ 9.7$

\vspace{1ex}
$\therefore$ Geometric Mean, \ $G \ = \ \sqrt[6]{3 \times 5 \times 7 \times 11 \times 14 \times 57} \ = \ 9.87$

\vspace{1ex}
$\therefore$ Harmonic Mean, \ $H \ = \ \frac{n}{\scalebox{1}{$\sum{\frac{1}{X}}$}} \ = \ \frac{6}{\scalebox{1}{$\left(\frac{1}{3} + \frac{1}{5} + \frac{1}{7} + \frac{1}{11} + \frac{1}{14} + \frac{1}{57}\right)$}} \ = \ \frac{6}{0.856} \ = \ 7.009$

\pagebreak
\textbf{\mred{Exr17.}} If the weights are 2, 1, 1, 3, 1, and 2 for the numbers 3, 5, 7, 11, 14, and 57
(exercise 16), compute the weighted average and variance.

\vspace{1ex}
\Heading{Solution}
\begin{center}
   \begin{tabularx}{\linewidth}{|C|C|C|C|}\hline
      $x$ & $w$ & $wx$ & $wx^2$\\\hline
      3 & 2 & 6 & 18\\\hline
      5 & 1 & 5 & 25\\\hline
      7 & 1 & 7 & 49\\\hline
      11 & 3 & 33 & 363\\\hline
      14 & 1 & 14 & 196\\\hline
      57 & 2 & 114 & 6498\\\hline
      \multicolumn{1}{r|}{$\sum=$} & 10 & 179 & 7149\\\cline{2-4}
   \end{tabularx}
\end{center}

\vspace{3ex}
$\therefore$ Weighted Average, \ $\bar{x} \ = \ \frac{\sum{wx}}{\sum{w}} \ = \ \frac{179}{10} \ = \ 17.9$

\vspace{3ex}
$\therefore$ Variance, \ $\sigma^2 \ = \ \frac{\sum{wx^2}}{N}-\left(\frac{\sum{wx}}{N}\right)^2 \ = \ \frac{7149}{10} - \left(\frac{179}{10}\right)^2 \ = \ 394.49$

\pagebreak
\Title{Regression}

\begin{tabular}{ll}
   \textbf{\mred{Q15.}} Calculate & $(i)$ the regression equation of x on y and y on x from the following data and\\
   & $(ii)$ estimate x when y = 20, \tabs $(iii)$ estimate y when x = 30.
\end{tabular}

\vspace{1ex}
\Heading{Solution}
\begin{center}
   \begin{tabularx}{\linewidth}{|C|C|C|C|C|C|C|}\hline
      $x$ & $x^{\prime}$ & $x^{\prime 2}$ & $y$ & $y^{\prime}$ & $y^{\prime 2}$ & $x^{\prime}y^{\prime}$ \\\hline
      10 & -4 & 16 & 5 & -3 & 9 & 12 \\\hline
      12 & -2 & 4 & 6 & -2 & 4 & 4 \\\hline
      13 & -1 & 1 & 7 & -1 & 1 & 1 \\\hline
      17 & 3 & 9 & 9 & 1 & 1 & 3 \\\hline
      18 & 4 & 16 & 13 & 5 & 25 & 20 \\\hline
      \mred{70} && \mred{46} && \mred{40} & \mred{40} & \mred{40}\\\hline
   \end{tabularx}
\end{center}

\vspace{3ex}
Now,

\vspace{1ex}
$\bar{x} = \frac{\sum{x}}{N} = \fracv[0]{70}{5}$
\tabs and \tabs $\bar{y} = \frac{\sum{y}}{N} = \fracv[0]{40}{5}$

\vspace{3ex}
$m = \frac{\sum{x^{\prime}y^{\prime}}}{\sum{x^{\prime 2}}} = \fracv[2]{40}{46}$
\tabs and \tabs $m^{\prime} = \frac{\sum{x^{\prime}y^{\prime}}}{\sum{y^{\prime 2}}} = \fracv[0]{40}{40}$
\vspace{2ex}

\vspace{3ex}
Regression equation,\\
\begin{tabular}{ll|l}
   $\begin{aligned}
      & (y-\bar{y}) = m(x-\bar{x})\\
      & y-8 = 0.87(x-14)\\
      & y = 0.87x -14 \times 0.87+8\\
      \therefore \ & y = 0.87x-4.18
   \end{aligned}$
   & \tabs &
   $\begin{aligned}
      & (x-\bar{x}) = m^{\prime}(y-\bar{y})\\
      & x-14 = 1(y-8)\\
      & x = y-8+14\\
      \therefore \ & x = y+6
   \end{aligned}$
\end{tabular}

\vspace{2ex}
$\begin{aligned}
   \text{when y = 20,} \tabs x \ &= (20) + 6 \ = 26\\
   \text{when x = 30,} \tabs y \ &= 0.87(30)-4.18 \ = 21.92
\end{aligned}$

\vspace{5ex}
Coefficient of linear correlation, $r \ = \ \frac{\sum{x^{\prime}y^{\prime}}}{\sqrt{\sum{x^{\prime 2}}\sum{y^{\prime 2}}}} \ = \ \frac{40}{\sqrt{(46)(40)}} \ = \ 0.933$
\end{document}
