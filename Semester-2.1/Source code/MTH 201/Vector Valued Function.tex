\documentclass[12pt]{article}
\usepackage[margin=1.27cm]{geometry}
\usepackage{setspace}
\usepackage{fontspec}
% \usepackage[T1]{fontenc}
% \usepackage[utf8]{inputenc}
\usepackage{amsmath,txfonts,amssymb,nicefrac,mathtools,pifont} %for math
\usepackage{array,tabularx,multirow,fmtcount} %for tables
\usepackage{tikz, pgfplots} %for diagram
\usepackage{multicol} %for multiple column
\usepackage{enumerate,enumitem,adjustbox} %for ordered list
\usepackage{graphicx,subcaption,wrapfig,tcolorbox} %for figure
\usepackage{xparse} %for commands & environments
\usepackage{lipsum} %miscellaneous
\usepackage{colortbl,xcolor,soul} %for default table & border

% #ANCHOR Font settings
\setmainfont{Oxygen}
\newfontfamily\banglafont[Script=Bengali]{Baloo Da 2}
\newfontfamily{\lstsansserif}{IBM Plex Mono}
\renewcommand{\normalsize}{\fontsize{11.5pt}{13pt}\selectfont}


\setlength{\arrayrulewidth}{0.35 pt}
\definecolor{border}{HTML}{A1A1AA}
\arrayrulecolor{border}


% #ANCHOR Document settings
\linespread{1.45}
\setlength\parindent{0pt}
\setlength\parskip{16pt}
\setlist[enumerate]{noitemsep}
\usetikzlibrary{shapes.geometric,decorations.pathreplacing,trees,arrows,positioning,shapes,fit,calc,decorations.markings, decorations.text}
\tikzset{every node/.append style={font=\footnotesize}}
\usepgfplotslibrary{fillbetween}
\pgfdeclarelayer{background}
\pgfsetlayers{background,main}
\pgfplotsset{compat=1.18}
\columnseprule=1pt
\everymath{\displaystyle}
% #ANCHOR Hypernation
\tolerance=1
\emergencystretch=\maxdimen
\hyphenpenalty=10000
\hbadness=10000
\newlength{\colWidth}



% #ANCHOR Colors
\definecolor{azure(colorwheel)}{rgb}{0.0, 0.5, 1.0}
\definecolor{carminepink}{rgb}{0.92, 0.3, 0.26}
\definecolor{orange}{rgb}{0.9, 0.55, 0.22}
\definecolor{violet}{rgb}{0.60, 0.45, 1}
% Syantax Highlighting Colors
\definecolor{keyword}{HTML}{D73A4A}
\definecolor{number}{HTML}{015CC5}
\definecolor{comment}{HTML}{6A737D}
\definecolor{string}{HTML}{1D825E}
\definecolor{function}{HTML}{743FD1}
\definecolor{orange}{HTML}{CF7842}
\definecolor{codeblack}{HTML}{24292F}
\definecolor{divider}{HTML}{A1A1AA}
\definecolor{border}{HTML}{D1D1D1}


% #ANCHOR Ordered & Unordered List
\setlist[itemize,1]{left=0cm, label={\textbullet}}
\setlist[itemize,2,3,4,5,6,7,8,9,10]{left=0.6cm, label={\textbullet}}
\setlist[enumerate,1]{left=0cm}
\setlist[enumerate,2,3,4,5,6,7,8,9,10]{left=0.6cm}
\setul{0.5ex}{0.125ex}



% #ANCHOR Colored Box
\let\oldul\ul
\renewcommand{\ul}[2][keyword]{\text{\setulcolor{#1}\oldul{#2}}}
\newcommand{\redbox}[1]{%
{\color{red}\fbox{\color{black}#1}}
}
\newcommand{\red}[1]{%
\textcolor{red}{#1}
}
\newcommand{\redeq}[1]{%
\text{\color{red}$#1$}
}
\newcommand{\mred}[1]{%
\textcolor{keyword}{#1}
}
\newcommand{\mredeq}[1]{%
\textcolor{keyword}{$#1$}
}
\newcommand{\blue}[1]{%
% {\color{number}#1\hspace{-0.4ex}}
\textcolor{number}{#1}
}
\newcommand{\blueeq}[1]{%
\text{\color{number}$#1$}
}
\newcommand{\cyanbox}[1]{%
{\color{teal}\fbox{\textcolor{black}{#1}}}
}
\newcommand{\cyan}[1]{%
\textcolor{teal}{#1}
}
\newcommand{\pink}[1]{%
\textcolor{magenta}{#1}
}
\newcommand{\orange}[1]{%
\textcolor{orange}{#1}
}
\newcommand{\violet}[1]{%
{\color{violet}#1}
}
\newcommand{\cyaneq}[1]{%
\text{\color{teal}$#1$}
}
\newcommand{\gray}[1]{%
\textcolor{comment}{#1}
}
\newcommand{\pinkeq}[1]{%
\text{\color{magenta}$#1$}
}
\renewcommand{\columnseprulecolor}{\color{divider}}




% #ANCHOR Tabular commands
\newcolumntype{P}[1]{>{\centering\arraybackslash}p{#1}}
\newcolumntype{M}[1]{>{\centering\arraybackslash}m{#1}}
\newcolumntype{C}{>{\centering\arraybackslash}X}
\newcommand{\rspan}[2]{\multirow{#1}{*}{#2}}
\newcommand{\thc}[1]{%
\multicolumn{1}{|c|}{\textbf{#1}}
}
\newcommand{\thcx}[1]{%
\multicolumn{1}{|C|}{\textbf{#1}}
}
\newcommand{\thl}[1]{%
\multicolumn{1}{|l|}{\textbf{#1}}
}
\newcommand{\thr}[1]{%
\multicolumn{1}{|r|}{\textbf{#1}}
}
% Adjusting arraystretch to modify vertical padding
\renewcommand{\arraystretch}{1.25}
% Adjusting tabcolsep to modify horizontal padding
\setlength{\tabcolsep}{10pt}



% #ANCHOR Math commands
\newcommand{\set}[1]{\{$#1$\}}
\newcommand{\tabs}{\ \ \ \ \ \ }
\newcommand{\tab}{\ \ \ }
\newcommand{\cmark}{\ding{51}}%
\newcommand{\xmark}{\ding{55}}%
\newcommand{\boldi}[1]{\boldsymbol{#1}}%
\newcommand{\wspace}{\ \ = \ \ }



% #ANCHOR New commands
\newcommand{\Title}[1]{%
   \begin{center}
      \textbf{\Large{#1}}
   \end{center}
}
\newcommand{\Heading}[1]{%
   \par\vspace{\dimexpr -\baselineskip + 16pt}
   {\fontsize{12pt}{13pt}\selectfont\textbf{#1}}
   \par\vspace{\dimexpr -\baselineskip + 6pt}
}
\newcommand{\BuleHeading}[1]{%
   \par\vspace{\dimexpr -\baselineskip + 16pt}
   {\fontsize{12pt}{13pt}\selectfont\textbf{\textcolor{number}{#1}}}
   \par\vspace{\dimexpr -\baselineskip + 6pt}
}
\newcommand{\CHeading}[1]{%
   \par\vspace{\dimexpr -\baselineskip + 16pt}
   \hspace{\fill}
   {\fontsize{12pt}{13pt}\selectfont\textbf{#1}}
   \hspace{\fill}
   \par\vspace{\dimexpr -\baselineskip + 6pt}
}
\newcommand{\Section}[1]{%
   \par\vspace{\dimexpr -\baselineskip + 16pt}
   \hspace{\fill}
   {\fontsize{13pt}{13pt}\selectfont\textbf{#1}}
   \hspace{\fill}
   \par\vspace{\dimexpr -\baselineskip + 6pt}
}
\newcommand{\seteqno}[1]{%
   \ \cdots \ \cdots \ \cdots \ (#1)
}
\newcommand{\eqor}{%
   \Rightarrow \ \ 
}
\newcommand{\tsub}[1]{%
\textsubscript{#1}\hspace{-0.45ex}
}
\newcommand{\tsup}[1]{%
\textsuperscript{#1}\hspace{-0.45ex}
}
\newcommand{\cbox}[2][cyan]{
\tikz\node[draw=#1,circle,inner sep=2pt,baseline=(a.base)](a){#2};
}
\newcommand{\hrline}{%
\vspace{1ex} {\color{gray}\hrule} \vspace{4ex}
}
\newcommand{\divideX}[1][divider]{{\hspace{1ex}\color{#1}{\vrule}\hspace{1ex}}}
\newcommand{\Reference}[2][Reference]{

\vspace{-0.5\baselineskip}
\begin{center}
   {\fontspec{Merriweather}\textbf{#1:} \textit{#2}} 
\end{center}
}
\newcommand{\bn}[1]{%
   {\banglafont #1}
}

\NewDocumentCommand{\Column}{O{0.49} O{1.5em} m m}{
   \setlength{\colWidth}{\linewidth-#1\linewidth-#2}
   \begin{minipage}[t]{#1\linewidth}
      \noindent
         #3
      \end{minipage}\hspace{\fill}{\color{divider}\vrule width 0.35pt}\hspace{\fill}
      \begin{minipage}[t]{\colWidth}
      \noindent
         #4
   \end{minipage}
}

% Vector commands
\renewcommand{\vec}[1]{\underline{\mathrm{#1}}}
\renewcommand{\r}{\mathrm{\textbf{r}}}
\renewcommand{\v}{\mathrm{v}}
\renewcommand{\a}{\mathrm{a}}
\let\oldkappa\kappa
\renewcommand{\kappa}{\scalebox{1.25}{$\oldkappa$}}
\newcommand{\vf}[2][t]{\vec{#2}(#1)}
\newcommand{\norm}[1]{\left\lVert\ #1\ \right\rVert}


\NewDocumentCommand{\dt}{O{t} m}{\vec{#2}^\prime(#1)}
\NewDocumentCommand{\vdt}{m}{\vec{#1}^\prime}
\NewDocumentCommand{\dtt}{O{t} m}{\vec{#2}^{\prime\prime}(#1)}
\NewDocumentCommand{\vdtt}{m}{\vec{#1}^{\prime\prime}}
\NewDocumentCommand{\vecf}{O{x} O{y} O{z}}{#1 \ \vec{i} + #2 \ \vec{j} + #3 \ \vec{k}}
\NewDocumentCommand{\vecxy}{O{x} O{y}}{#1 \ \vec{i} + #2 \ \vec{j}}
\NewDocumentCommand{\vecbf}{O{x} O{y} O{z}}{\left(#1,\ #2,\ #3\right)}
\begin{document}
\Title{Vector Functions and Space Curves}
\textbf{\mred{Exr-1}:} Find the parametric equation of a tangent line of the circular helix at $t=\frac{\pi}{4}$.
\vspace{-\baselineskip}
$$x=\cos{t}, \tabs y=\sin{t}, \tabs z=t$$

\vspace{-\baselineskip}
\Heading{Solution:}

\vspace{1ex}
$\begin{aligned}
   \text{Let,\ } \vf{\r} \ & = \ \vecf \ = \ \vecf[\cos{t}][\sin{t}][t]\\
   \therefore \dt{\r} \ & = \ \vecf[-\sin][\cos][1]\\[1ex]
\end{aligned}$

\vspace{1ex}
Now we have,\\[1ex]
$\begin{aligned}
   \therefore \vf[t_o]{\r} \ = \ \vec{\r}\left(\frac{\pi}{4}\right) & \ = \ \vecf[\frac{1}{\sqrt{2}}][\frac{1}{\sqrt{2}}][\frac{\pi}{4}]\\[1ex]
   \therefore \dt[t_o]{\r} \ = \ \vec{\r}^\prime\left(\frac{\pi}{4}\right) & \ = \ -\vecf[\frac{1}{\sqrt{2}}][\frac{1}{\sqrt{2}}][1]\\
\end{aligned}$

\vspace{3ex}
Equation of tangent line is,\\
$\begin{aligned}
   \vf{\r} & \ = \ \vec{\r}(t_o) + t\ \dt[t_o]{\r}\\
   & \ = \ \left(\vecf[\frac{1}{\sqrt{2}}][\frac{1}{\sqrt{2}}][\frac{\pi}{4}]\right) + t\ \left(-\vecf[\frac{1}{\sqrt{2}}][\frac{1}{\sqrt{2}}][1]\right)\\[1ex]
   & \ = \ \vecf[\left(\frac{1}{\sqrt{2}}-\frac{t}{\sqrt{2}}\right)][\left(\frac{1}{\sqrt{2}}+\frac{t}{\sqrt{2}}\right)][\left(\frac{\pi}{4}+t\right)]
\end{aligned}$

\vspace{3ex}
The parametric equations are,\\[1ex]
$\begin{aligned}
   \vf{x} \ &= \ \frac{1}{\sqrt{2}}-\frac{t}{\sqrt{2}}\\[1ex]
   \vf{y} \ &= \ \frac{1}{\sqrt{2}}+\frac{t}{\sqrt{2}}\\[1ex]
   \vf{z} \ &= \ \frac{\pi}{4}+t
\end{aligned}$

\pagebreak
\textbf{\mred{Exr-2}:} Find the curvature for the circular helix \vspace{-\baselineskip}
$$x=a\cos{t}, \tabs y=a\sin{t}, \tabs z=ct$$

\vspace{-\baselineskip}
\Heading{Solution:}
\vspace{1ex}
$\begin{aligned}
   \text{Let,\ } \vec{\r} \ & = \ \vecbf \ = \ \vecbf[a\cos{t}][a\sin{t}][ct]\\
   \therefore \vdt{\r} \ & = \ \vecbf[-a\sin{t}][a\cos{t}][c]\\
   \therefore \vdtt{\r} \ & = \ \vecbf[-a\cos{t}][-a\sin{t}][0]\\
   \therefore \norm{\vdt{\r}} \ & = \ \sqrt{a^2 \sin ^2 t+a^2 \cos ^2 t+c^2} \ = \ \sqrt{a^2+c^2} \\[1ex]
   \therefore \vdt{\r} \times \vdtt{\r} \ & = \
   \left|\begin{array}{ccc}
      \vec{i} & \vec{j} & \vec{k} \\
      -a\sin{t} & a\cos{t} & c \\
      -a\cos{t} & -a\sin{t} & 0
      \end{array}\right|\\
   \ & = \ \vecf[(ac\sin{t})][(ac\cos{t})][(a^2 \sin^2{t} + a^2 \cos^2{t})]\\
   \ & = \ \vecf[(ac\sin{t})][(ac\cos{t})][a^2]\\[1ex]
   \therefore \norm{\vdt{\r} \times \vdtt{\r}} \ & = \sqrt{a^2 c^2 \sin ^2 t+a^2 c^2 \cos ^2 t+a^4} \ = \ a\sqrt{a^2+c^2}\\[1ex]
   \vf{\kappa} \ &= \ \frac{\norm{\vdt{\r} \times \vdtt{\r}}}{\norm{\vdt{\r}}^3} \ =\frac{a \sqrt{a^2+c^2}}{\left(\sqrt{a^2+c^2}\right)^3} = \frac{a}{a^2+c^2}\\
\end{aligned}$


\pagebreak
\textbf{\mred{Exr-3}:} The graph of the ellipse is given by
\vspace{-\baselineskip}
$$\vec{\r} = \vecxy[2\cos{t}][3\sin{t}] \tabs 0 \leq t \leq 2\pi$$

\vspace{-\baselineskip}
Find the curvature of the ellipse at the end point of the major and minor axes.

\vspace{2ex}
\Heading{Solution:}
\vspace{1ex}
% $\vec{\r}=2 \cos t \underline{i}+3 \sin t \underline{j}$
$\begin{aligned}
\text{Given,}\ \vec{\r} & \ = \ \vecxy[(2\cos{t})][(3\sin{t})]\\
\vdt{\r} & \ = \ \vecxy[(-2\sin{t})][(3\cos{t})] \\
\vdtt{\r} & \ = \ \vecxy[(-2\cos{t})][(-3\sin{t})] \\
\norm{\vdt{\r}} & \ = \ \sqrt{4\sin^2{t}+9\cos^2{t}}\\[1ex]
\therefore \vdt{\r} \times \vdtt{\r} \ & = \
\left|\begin{array}{ccc}
   \vec{i} & \vec{j} & \vec{k} \\
   -2\sin{t} & 3\cos{t} & 0 \\
   -2\cos{t} & -3\sin{t} & 0
   \end{array}\right|\\
   & \ = \ i(0) - j(0) + \vec{k}\left(6\sin^2{t} + 6\cos^2{t}\right) \ = \ 6\vec{k}\\
   \therefore \norm{\vdt{\r} \times \vdtt{\r}} \ & = \sqrt{6^2} \ = \ 6\\[2ex]
   \vf{\kappa} \ &= \ \frac{\norm{\vdt{\r} \times \vdtt{\r}}}{\norm{\vdt{\r}}^3} \ =\frac{6}{\left(\sqrt{4\sin^2{t}+9\cos^2{t}}\right)^3} \quad \seteqno{1}\\
\end{aligned}$

\vspace{4ex}
The end point of the minor axis are $(2,0)$ and $(-2,0)$ corresponds to, $t=0$ and $t=\pi$\\
Now put $t=0$ and $t=\pi$ in $(1)$\\[1ex]
$\Rightarrow \ \kappa(0)=\frac{6}{(\sqrt{0+9})^3}=\frac{6}{27}=\frac{2}{9}$ \tabs and \tabs $\kappa(\pi)=\frac{2}{9}$

\vspace{4ex}
Similarly the end point of the major anis are $(0,3)$ and $(0,-3)$ corresponds to $t=\frac{\pi}{2}$ and $t=\frac{3\pi}{2}$\\
Now put $t=\frac{\pi}{2}$ and $t = \frac{3\pi}{2}$ in $(1)$,\\[1ex]
$\Rightarrow \ \kappa\left(\frac{\pi}{2}\right)=\frac{6}{(\sqrt{4+0})^3}=\frac{6}{8}=\frac{3}{4}$  \tabs and \tabs $\kappa\left(\frac{3 \pi}{2}\right)=\frac{3}{4}$


\pagebreak
\textbf{\mred{Exr-4}:} Find the arc length of the circular helix
\vspace{-\baselineskip}
$$x=a\cos{t}, \tabs y=a\sin{t}, \tabs z=ct \tabs \quad \text{where} \tab 0 \leq t \leq \pi$$

\vspace{-\baselineskip}
\Heading{Solution:}
\vspace{1ex}
$\begin{aligned}
   \text{Let,\ } \vec{\r} \ & = \ \vecbf \ = \ \vecbf[a\cos{t}][a\sin{t}][ct]\\[1ex]
   \therefore \frac{d\r}{dt} \ & = \ \vecbf[-a\sin{t}][a\cos{t}][c]\\[1ex]
   \therefore \norm{\frac{d\r}{dt}} \ & = \ \sqrt{a^2 \sin ^2 t+a^2 \cos ^2 t+c^2} \ = \ \sqrt{a^2+c^2}
\end{aligned}$

\vspace{3ex}
$\begin{aligned}
   \therefore L & = \int_{a}^{b} \norm{\frac{d\r}{dt}} \,dt
   \wspace \int_{0}^{\pi} \sqrt{a^2+c^2}\ dt
   \wspace \sqrt{a^2+c^2}\int_{0}^{\pi} 1\ dt
   \wspace \sqrt{a^2+c^2} \left[\ t\ \right]_{0}^{\pi}
   \wspace \pi\sqrt{a^2+c^2}
\end{aligned}$

\vspace{20ex}
\textbf{\mred{Exr-5}:} Find the arc length of the curve

\vspace{-\baselineskip}
$$\vf{\r} \ = \ \vecf[e^t\cos{t}][e^t\sin{t}][e^t] \tabs \quad \text{where} \tab 0 \leq t \leq \frac{\pi}{2}$$

\vspace{-\baselineskip}
\Heading{Solution:}
\vspace{1ex}
$\begin{aligned}
   \text{Let,\ } \vec{\r} \ & = \ \vecbf \ = \ \vecbf[e^t\cos{t}][e^t\sin{t}][e^t]\\[1ex]
   \therefore \frac{d\r}{dt} \ & = \ \vecf[(e^t\cos{t}-e^t\sin{t})][(e^t\sin{t}+e^t\cos{t})][(e^t)]\\[1ex]
   \therefore \norm{\frac{d\r}{dt}} \ & = \ \sqrt{e^{2t}\left(\cos^2{t}+\sin^2{t}-2\sin{t}\cos{t}\right) + e^{2t}\left(\sin^2{t}+\cos^2{t}+2\sin{t}\cos{t}\right) + e^{2t}}\\[1ex]
   & = \ \sqrt{e^{2t}\left(\cos^2{t}+\sin^2{t}-2\sin{t}\cos{t} + \sin^2{t}+\cos^2{t}+2\sin{t}\cos{t} + 1\right)}\\[1ex]
   & = \ \sqrt{e^{2t}\left(1 + 1 + 1\right)} \ = \ \sqrt{3}e^t\\[1ex]
\end{aligned}$

\vspace{3ex}
$\begin{aligned}
   \therefore L & = \int_{a}^{b} \norm{\frac{d\r}{dt}} \,dt
   \wspace \int_{0}^{\frac{\pi}{2}} \sqrt{3}e^t\ dt
   \wspace \sqrt{3}\int_{0}^{\frac{\pi}{2}} e^t\ dt
   \wspace \sqrt{3} \left[\ e^t\ \right]_{0}^{\frac{\pi}{2}}
   \wspace \sqrt{3} \left(e^{\frac{\pi}{2}}-1\right)
\end{aligned}$


% \pagebreak
% For a particle moving along a curve din 2-space or 3-space, The velocity and acceleration vector
% can be written as
% $$
% \begin{aligned}
% & \vec{v} \ = \ \frac{d s}{d t} \ \vec{T} \tab \seteqno{1} \\
% & \vec{a} \ = \ \frac{d^2s}{d t^2} \ \vec{T} + \kappa\left(\frac{d s}{d t}\right)^2 \vec{N} \tab \seteqno{2}
% \end{aligned}
% $$

% \begin{tabular}{rl}
%    where, & $s$ \ is the  arc length parameter,\\[-0.5ex]
%    & $\vec{T}$ \ is a unit tangent vector \\[-0.5ex]
%    & $\vec{N}$ \ is a unit normal vector \\[-0.5ex]
%    & $\kappa$ \ is curvature \\
% \end{tabular}

% \vspace{3ex}
% The coefficient of $\vec{T}$ and $\vec{N}$ in $(2)$,

% \vspace{1ex}
% $\begin{aligned}
% & a_T \ = \ \frac{d^2s}{d t^2} \quad \text{is called the scalar tangential component of acceleration.} \\
% & a_N \ = \ \kappa\left(\frac{d s}{d t}\right)^2 \text{is called the scaler normal component of acceleration.}\\
% \end{aligned}$


% \vspace{3ex}
% This provides useful insight into the physical properties of the tangential and normal components acceleration. They are not always the best formulas use for computation.
% We shall now derive some more useful formulas that express the scalar normal and tangential components of acceleration directly, in terms of velocity and acceleration of the particle.


% $$\begin{aligned}
%    a_T \ = \ \frac{\vec{v} \cdot\vec{a}}{\|\vec{v}\|}, \quad a_N \ = \ \frac{\|\vec{v} \times \vec{a}\|}{\|\vec{v}\|}\\
%    \text{curvature } \kappa \ = \ \frac{\|\vec{v} \times \vec{a}\|}{\|\vec{v}\|^3}
% \end{aligned}$$

\pagebreak
\textbf{\mred{Exr-1}:} Suppose a particle moves through 3-space and its position vector at time $t$ is

\vspace{-\baselineskip}
$$\vec{r}(t)=t\ \vec{i} + t^2 \ \vec{j} + t^3 \ \vec{k}$$

\vspace{-\baselineskip}
\begin{enumerate}[label=(\alph*)]
   \item find the scalar tangential and normal components of acceleration at time \ $t$
   \item find the scalar tangential and normal components of acceleration at time \ $t=1$
   \item find the vector tangential and normal components of acceleration at time \ $t=1$.
   \item find the curvature of the path at the point where the particle is located at time \ $t=1$
\end{enumerate}

\Heading{Solution:}
\begin{enumerate}[label=\textbf{\mred{(\alph*)}}]
\item 
\vspace{1ex}
$\begin{aligned}[t]
\text{ Given } \quad \vec{r}(t) \ &= \ t\ \vec{i} + t^2 \ \vec{j} + t^3 \ \vec{k}\\
\therefore \vec{v} \ &= \ \frac{d r}{d t} \ = \ \vec{r}^{\prime} \ = \ 1 \ \vec{i}+2 t \ \vec{j}+3 t^2\ \vec{k} \\
\therefore\vec{a} \ &= \ \vec{v}^{\prime}(t) \ = \ 0\ \vec{i}+2 \vec{j} + 6t \ \vec{k} \\
\end{aligned}$

\vspace{3ex}
Now,\\
$\begin{aligned}
  \norm{v} \ = \ & \sqrt{1+4 t^2+9 t^4} \\
   \therefore \ \vec{v} \cdot \vec{a} \ = \ & 0+4 t+18 t^3 \\
   \therefore \ a_T
   \ = \ & \frac{\vec{v} \cdot\vec{a}}{\norm{v}}
   \ = \ -\frac{4 t+18 t^3}{\sqrt{1+4 t^2+9 t^4}}
\end{aligned}$

\vspace{3ex}
$\begin{aligned}
\therefore \ \vec{v} \times a \ &= \ \left|\begin{array}{ccc}
\vec{i} & \vec{j} & \vec{j} \\
1 & 2 t & 3 t^2 \\
0 & 2 & 6 t
\end{array}\right| \ = \ \vec{i} \left(12 t^2-6 t^2\right) - \vec{j}(6 t-0) + \vec{k}(2-0) \ = \ 6 t^2 \ \vec{i}-6 t \ \vec{j} + 2 \ \vec{k}\\[1ex]
\therefore \ a_N \ &= \ \frac{\|v \times\vec{a}\|}{\|\vec{v}\|}
\ = \ \frac{\sqrt{36 t^4+36 t^2+4}}{\sqrt{1++4 t^2+9 t^4}}
\end{aligned}$


\vspace{4ex}
\item
$\begin{aligned}[t]
\text{At $t=1$, \ }& a_T \ = \ \frac{4+18}{\sqrt{14}} = \frac{22}{\sqrt{14}} \\[0.5ex]
& a_N \ = \ \frac{\sqrt{36+36+4}}{\sqrt{14}}= \frac{\sqrt{76}}{\sqrt{14}}
\end{aligned}$

\pagebreak
\item
Since \ $a_T \vec{T}$ \ is the vector tangential component of accelaration where \ $\vec{T}= \frac{\vec{v}}{\norm{\vec{v}}}$

$\begin{aligned}
\text{At } \ t=1, \quad  \vec{T}(1)
\ &= \ \frac{v(1)}{\norm{v(1)}}
\ = \ \frac{1 \ \vec{i}+2 t  \ \vec{j}+3 t^2 \ \vec{k}}{\sqrt{1+4+9}}
\ = \ \frac{1}{\sqrt{14}}(\vec{i}+2 \dot{J}+3\vec{k})\\[1ex]
\therefore A_T \vec{T}(1)
\ &= \ \frac{22}{\sqrt{14}} \vec{T}(1) \ = \ \frac{22}{\sqrt{14}} \cdot \frac{1}{\sqrt{14}}(1\vec{i}+2 \dot{J}+3\vec{k}) \\[1ex]
& =\frac{11}{7}(i+2\vec{i}+3\vec{k})
\ = \ \frac{11}{7} \vec{i}+\frac{22}{7} \vec{j}+\frac{33}{7} \vec{k}
\end{aligned}$


\vspace{3ex}
Now,\\
$\begin{aligned}
   \vec{a} \ &= \ a_T \Vec{T} + a_N \vec{N} \\
   a_N \vec{N} \ &= \ \vec{a} - a_T \Vec{T} \\
\end{aligned}$

\vspace{1ex}
$\begin{aligned}
\text { At } t=1, \quad a_N N(1) \ &= \ \vec{a}(1)-a_T T(1) \\
&=(2 \vec{j} + 6 \vec{k}) -\left(\frac{11}{7} \vec{i}+\frac{22}{7} \vec{j}+\frac{33}{7}\vec{k}\right) \\
 &=-\frac{11}{7} \vec{i} - \frac{8}{7} \vec{j} + \frac{9}{7}\vec{k} \\
\end{aligned}$

\vspace{3ex}
\item
At \ $t=1$, \quad $\norm{\vec{v} \times\vec{a}} \ = \ \sqrt{36+36+4}=\sqrt{33} \sqrt{76}=2 \sqrt{19}$

\vspace{1ex}
$\norm{v} \ = \ \sqrt{1+4+9} \ = \ \sqrt{14}$

\vspace{1ex}
$\therefore$ curvature $\kappa \ = \ \frac{\norm{\vec{v} \times \vec{a}}}{\norm{\vec{v}}^3} \ = \ \frac{2 \sqrt{19}}{(\sqrt{14})^3}$

\end{enumerate}
\end{document}