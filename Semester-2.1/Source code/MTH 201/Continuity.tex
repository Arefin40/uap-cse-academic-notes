\documentclass[12pt]{article}
\usepackage[margin=1.27cm]{geometry}
\usepackage{setspace}
\usepackage{fontspec}
% \usepackage[T1]{fontenc}
% \usepackage[utf8]{inputenc}
\usepackage{amsmath,txfonts,amssymb,nicefrac,mathtools,pifont} %for math
\usepackage{array,tabularx,multirow,fmtcount} %for tables
\usepackage{tikz, pgfplots} %for diagram
\usepackage{multicol} %for multiple column
\usepackage{enumerate,enumitem,adjustbox} %for ordered list
\usepackage{graphicx,subcaption,wrapfig,tcolorbox} %for figure
\usepackage{xparse} %for commands & environments
\usepackage{lipsum} %miscellaneous
\usepackage{colortbl,xcolor,soul} %for default table & border

% #ANCHOR Font settings
\setmainfont{Oxygen}
\newfontfamily\banglafont[Script=Bengali]{Baloo Da 2}
\newfontfamily{\lstsansserif}{IBM Plex Mono}
\renewcommand{\normalsize}{\fontsize{11.5pt}{13pt}\selectfont}


\setlength{\arrayrulewidth}{0.35 pt}
\definecolor{border}{HTML}{A1A1AA}
\arrayrulecolor{border}


% #ANCHOR Document settings
\linespread{1.45}
\setlength\parindent{0pt}
\setlength\parskip{16pt}
\setlist[enumerate]{noitemsep}
\usetikzlibrary{shapes.geometric,decorations.pathreplacing,trees,arrows,positioning,shapes,fit,calc,decorations.markings, decorations.text}
\tikzset{every node/.append style={font=\footnotesize}}
\usepgfplotslibrary{fillbetween}
\pgfdeclarelayer{background}
\pgfsetlayers{background,main}
\pgfplotsset{compat=1.18}
\columnseprule=1pt
\everymath{\displaystyle}
% #ANCHOR Hypernation
\tolerance=1
\emergencystretch=\maxdimen
\hyphenpenalty=10000
\hbadness=10000
\newlength{\colWidth}



% #ANCHOR Colors
\definecolor{azure(colorwheel)}{rgb}{0.0, 0.5, 1.0}
\definecolor{carminepink}{rgb}{0.92, 0.3, 0.26}
\definecolor{orange}{rgb}{0.9, 0.55, 0.22}
\definecolor{violet}{rgb}{0.60, 0.45, 1}
% Syantax Highlighting Colors
\definecolor{keyword}{HTML}{D73A4A}
\definecolor{number}{HTML}{015CC5}
\definecolor{comment}{HTML}{6A737D}
\definecolor{string}{HTML}{1D825E}
\definecolor{function}{HTML}{743FD1}
\definecolor{orange}{HTML}{CF7842}
\definecolor{codeblack}{HTML}{24292F}
\definecolor{divider}{HTML}{A1A1AA}
\definecolor{border}{HTML}{D1D1D1}


% #ANCHOR Ordered & Unordered List
\setlist[itemize,1]{left=0cm, label={\textbullet}}
\setlist[itemize,2,3,4,5,6,7,8,9,10]{left=0.6cm, label={\textbullet}}
\setlist[enumerate,1]{left=0cm}
\setlist[enumerate,2,3,4,5,6,7,8,9,10]{left=0.6cm}
\setul{0.5ex}{0.125ex}



% #ANCHOR Colored Box
\let\oldul\ul
\renewcommand{\ul}[2][keyword]{\text{\setulcolor{#1}\oldul{#2}}}
\newcommand{\redbox}[1]{%
{\color{red}\fbox{\color{black}#1}}
}
\newcommand{\red}[1]{%
\textcolor{red}{#1}
}
\newcommand{\redeq}[1]{%
\text{\color{red}$#1$}
}
\newcommand{\mred}[1]{%
\textcolor{keyword}{#1}
}
\newcommand{\mredeq}[1]{%
\textcolor{keyword}{$#1$}
}
\newcommand{\blue}[1]{%
% {\color{number}#1\hspace{-0.4ex}}
\textcolor{number}{#1}
}
\newcommand{\blueeq}[1]{%
\text{\color{number}$#1$}
}
\newcommand{\cyanbox}[1]{%
{\color{teal}\fbox{\textcolor{black}{#1}}}
}
\newcommand{\cyan}[1]{%
\textcolor{teal}{#1}
}
\newcommand{\pink}[1]{%
\textcolor{magenta}{#1}
}
\newcommand{\orange}[1]{%
\textcolor{orange}{#1}
}
\newcommand{\violet}[1]{%
{\color{violet}#1}
}
\newcommand{\cyaneq}[1]{%
\text{\color{teal}$#1$}
}
\newcommand{\gray}[1]{%
\textcolor{comment}{#1}
}
\newcommand{\pinkeq}[1]{%
\text{\color{magenta}$#1$}
}
\renewcommand{\columnseprulecolor}{\color{divider}}




% #ANCHOR Tabular commands
\newcolumntype{P}[1]{>{\centering\arraybackslash}p{#1}}
\newcolumntype{M}[1]{>{\centering\arraybackslash}m{#1}}
\newcolumntype{C}{>{\centering\arraybackslash}X}
\newcommand{\rspan}[2]{\multirow{#1}{*}{#2}}
\newcommand{\thc}[1]{%
\multicolumn{1}{|c|}{\textbf{#1}}
}
\newcommand{\thcx}[1]{%
\multicolumn{1}{|C|}{\textbf{#1}}
}
\newcommand{\thl}[1]{%
\multicolumn{1}{|l|}{\textbf{#1}}
}
\newcommand{\thr}[1]{%
\multicolumn{1}{|r|}{\textbf{#1}}
}
% Adjusting arraystretch to modify vertical padding
\renewcommand{\arraystretch}{1.25}
% Adjusting tabcolsep to modify horizontal padding
\setlength{\tabcolsep}{10pt}



% #ANCHOR Math commands
\newcommand{\set}[1]{\{$#1$\}}
\newcommand{\tabs}{\ \ \ \ \ \ }
\newcommand{\tab}{\ \ \ }
\newcommand{\cmark}{\ding{51}}%
\newcommand{\xmark}{\ding{55}}%
\newcommand{\boldi}[1]{\boldsymbol{#1}}%
\newcommand{\wspace}{\ \ = \ \ }



% #ANCHOR New commands
\newcommand{\Title}[1]{%
   \begin{center}
      \textbf{\Large{#1}}
   \end{center}
}
\newcommand{\Heading}[1]{%
   \par\vspace{\dimexpr -\baselineskip + 16pt}
   {\fontsize{12pt}{13pt}\selectfont\textbf{#1}}
   \par\vspace{\dimexpr -\baselineskip + 6pt}
}
\newcommand{\BuleHeading}[1]{%
   \par\vspace{\dimexpr -\baselineskip + 16pt}
   {\fontsize{12pt}{13pt}\selectfont\textbf{\textcolor{number}{#1}}}
   \par\vspace{\dimexpr -\baselineskip + 6pt}
}
\newcommand{\CHeading}[1]{%
   \par\vspace{\dimexpr -\baselineskip + 16pt}
   \hspace{\fill}
   {\fontsize{12pt}{13pt}\selectfont\textbf{#1}}
   \hspace{\fill}
   \par\vspace{\dimexpr -\baselineskip + 6pt}
}
\newcommand{\Section}[1]{%
   \par\vspace{\dimexpr -\baselineskip + 16pt}
   \hspace{\fill}
   {\fontsize{13pt}{13pt}\selectfont\textbf{#1}}
   \hspace{\fill}
   \par\vspace{\dimexpr -\baselineskip + 6pt}
}
\newcommand{\seteqno}[1]{%
   \ \cdots \ \cdots \ \cdots \ (#1)
}
\newcommand{\eqor}{%
   \Rightarrow \ \ 
}
\newcommand{\tsub}[1]{%
\textsubscript{#1}\hspace{-0.45ex}
}
\newcommand{\tsup}[1]{%
\textsuperscript{#1}\hspace{-0.45ex}
}
\newcommand{\cbox}[2][cyan]{
\tikz\node[draw=#1,circle,inner sep=2pt,baseline=(a.base)](a){#2};
}
\newcommand{\hrline}{%
\vspace{1ex} {\color{gray}\hrule} \vspace{4ex}
}
\newcommand{\divideX}[1][divider]{{\hspace{1ex}\color{#1}{\vrule}\hspace{1ex}}}
\newcommand{\Reference}[2][Reference]{

\vspace{-0.5\baselineskip}
\begin{center}
   {\fontspec{Merriweather}\textbf{#1:} \textit{#2}} 
\end{center}
}
\newcommand{\bn}[1]{%
   {\banglafont #1}
}

\NewDocumentCommand{\Column}{O{0.49} O{1.5em} m m}{
   \setlength{\colWidth}{\linewidth-#1\linewidth-#2}
   \begin{minipage}[t]{#1\linewidth}
      \noindent
         #3
      \end{minipage}\hspace{\fill}{\color{divider}\vrule width 0.35pt}\hspace{\fill}
      \begin{minipage}[t]{\colWidth}
      \noindent
         #4
   \end{minipage}
}

% Vector commands
\renewcommand{\vec}[1]{\underline{\mathrm{#1}}}
\renewcommand{\r}{\mathrm{\textbf{r}}}
\renewcommand{\v}{\mathrm{v}}
\renewcommand{\a}{\mathrm{a}}
\renewcommand{\lim}{\operatorname*{Lt}}
\newcommand{\vf}[2][t]{\vec{#2}(#1)}
\newcommand{\norm}[1]{\left\lVert\ #1\ \right\rVert}


\NewDocumentCommand{\Lt}{O{(0, 0)} O{(x,y)}}{\lim_{#2 \to #1}}
\NewDocumentCommand{\dt}{O{t} m}{\vec{#2}^\prime(#1)}
\NewDocumentCommand{\vdt}{m}{\vec{#1}^\prime}
\NewDocumentCommand{\dtt}{O{t} m}{\vec{#2}^{\prime\prime}(#1)}
\NewDocumentCommand{\vdtt}{m}{\vec{#1}^{\prime\prime}}
\NewDocumentCommand{\vecf}{O{x} O{y} O{z}}{#1 \ \vec{i} + #2 \ \vec{j} + #3 \ \vec{k}}
\NewDocumentCommand{\vecxy}{O{x} O{y}}{#1 \ \vec{i} + #2 \ \vec{j}}
\NewDocumentCommand{\vecbf}{O{x} O{y} O{z}}{\left(#1,\ #2,\ #3\right)}
\begin{document}
\Title{Continuity}

\Heading{Definition Of Continuity}
A function $f(x,y)$ is said to be continuous at $(x_o, y_o)$ provided the following conditions are satisfied:

\vspace{-0.5\baselineskip}
\begin{enumerate}
   \item $f(x_o, y_o)$ is defined.
   \item $\lim_{(x,y) \to (x_o, y_o)} f(x, y)$ exists.
   \item $\lim_{(x,y) \to (x_o, y_o)} f(x, y) = f(x_o, y_o)$.
\end{enumerate}

\vspace{1ex}
\Title{Mathematical Problems}
\textbf{\mred{Exr1.}} Test the continuity for $f(x,y)$ at $(x,y)=(0,0)$ where,

\vspace{-0.5\baselineskip}
\begin{equation*}
   f(x,y) =
   \begin{cases}
      \frac{2xy}{x^2+y^2}, & when \ (x,y) \neq (0,0)\\
           0, & when \ (x,y) = (0,0)\\
   \end{cases}
\end{equation*}

\vspace{-\baselineskip}
\Heading{Solution:}

\vspace{1ex}
$\begin{aligned}
   &\text{At,\ } (x,y) = (0,0) \quad f(x,y)=0 \quad \text{defined.}\\
   &\text{At,\ } (x,y) \neq (0,0) \quad f(x,y)=\frac{2xy}{x^2+y^2}\\
\end{aligned}$

\vspace{2ex}
Now,\\
$\begin{aligned}
   \Lt f(x, y) & = \Lt \frac{2xy}{x^2+y^2} \\
   & = \Lt \frac{2xmx}{x^2+m^2x^2} \tabs [\text{along } y=mx] \\
   & = \Lt \frac{x^2\ 2m}{x^2(1+m^2)} \ = \ \Lt \frac{2m}{(1+m^2)}  \\
\end{aligned}$

\vspace{3ex}
Since, this limit changes with the value of $m$. So, there is no single value of $m$.\\
Hence, $\Lt f(x, y)$ \ does not exist.\\
$\therefore f(x,y)$ is not continuous at $(0,0)$.

\vspace{3ex}
\textbf{\mred{Exr2.}} Test the continuity for $f(x,y)$ at $(x,y)=(0,0)$ where,

\vspace{-0.5\baselineskip}
\begin{equation*}
   f(x,y) =
   \begin{cases}
      xy \ln{(x^2+y^2)}, & when \ (x,y) \neq (0,0)\\
           0, & when \ (x,y) = (0,0)\\
   \end{cases}
\end{equation*}

\vspace{-\baselineskip}
\Heading{Solution:}

\vspace{1ex}
$\begin{aligned}
   &\text{At,\ } (x,y) = (0,0) \quad f(x,y)=0 \quad \text{defined.}\\
   &\text{At,\ } (x,y) \neq (0,0) \quad f(x,y)=xy \ln{(x^2+y^2)}\\
\end{aligned}$

\vspace{2ex}
Now,\\
\begin{minipage}[t]{0.6\linewidth}
\noindent
$\begin{aligned}
   \Lt f(x, y) & = \Lt xy \ln{(x^2+y^2)} \\
   & =\Lt[0^+][r] r\cos{\theta} \cdot r\sin{\theta} \cdot \ln{r^2} \\
   & =\Lt[0^+][r] r^2 \cdot \frac{\sin{2\theta}}{2} \cdot \ln{r^2} \seteqno{1} \\
\end{aligned}$
\end{minipage}\hspace{0.5ex}{\vrule width 1pt}\hspace{0.5ex}
\begin{minipage}[t]{0.38\linewidth}
   \vspace{-2\baselineskip}
   using polar coordinates,\\
   $\begin{aligned}
      x &= r\cos{\theta}\\[-1ex]
      y &= r\sin{\theta}\\[-1ex]
      r &= \sqrt{x^2+y^2}
   \end{aligned}$

   \vspace{1ex}
   when $(x,y) = (0,0)$, then $r \to 0^+$.
\end{minipage}

\vspace{2ex}
Since: \ $\sin{2\theta \leq 1}$,\\
$\left|xy \ln{(x^2+y^2)}\right| \ = \ \left|\frac{r^2 \sin{2\theta}\ \ln{r^2}}{2}\right| \ = \ \left|\frac{r^2 \ln{r^2}}{2}\right|$

\vspace{2ex}
from $(1)$,\\
$\begin{aligned}
   \Lt[0][r] \frac{r^2 \ln{r^2}}{2} \quad = \
   \Lt[0][r] \frac{\ln{r^2}}{\scalebox{1}{$\frac{2}{r^2}$}}   \quad = \
   \Lt[0][r] \frac{\scalebox{1}{$\frac{1}{r^2}$} 
   \ 2r}{\scalebox{1}{$\frac{-4}{r^3}$}} \quad = \
   \Lt[0][r] \frac{-r^2}{2} \quad = \ 0
\end{aligned}$


\vspace{3ex}
Since, $\Lt f(x, y)$ \ exist \ and \ also, $\Lt f(x, y) = f(0,0)$\\[1ex]
$\therefore f(x,y)$ is continuous at $(0,0)$.

\vspace{5ex}
\textbf{\mred{Exr3.}} Show that the function \ $f(x,y) = \frac{2x^2y}{x^4+y^2}$ \ has
no limit as $(x,y)$ approaches $(0,0)$.


\Heading{Solution:}
Along the curve $y=mx^2, x\neq 0$, the function,\\[1ex]
$f(x,y)|_{y=mx^2}
\ = \ \frac{2x^2(mx^2)}{x^4+(mx^2)^2}
\ = \ \frac{2mx^4}{x^4(1+m^2)}
\ = \ \frac{2m}{1+m^2}$

\vspace{3ex}
$\Lt f(x,y) \ = \ \Lt f(x,y)|_{y=mx^2}
\ = \ \frac{2m}{1+m^2} = constant$

\vspace{3ex}
This limit  varies with the path of approaches.

If $(x,y)$ approaches $(0,0)$ along the parabola $y=mx^2$ for $(m=1)$ and the limit is $1$ and

if $(x,y)$ approaches $(0,0)$ along the x-axis $y=0$ for $(m=0)$ and the limit is $0$.

By two path test $f$ has no limit as $(x,y)$ approaches $(0,0)$.



% \vspace{3ex}
\pagebreak
\textbf{\mred{Exr4.}} Test the continuity for $f(x,y)$ at $(x,y)=(0,0)$ where,

\vspace{-0.5\baselineskip}
\begin{equation*}
   f(x,y) =
   \begin{cases}
      \frac{\sin{(x^2+y^2)}}{x^2+y^2}, & when \ (x,y) \neq (0,0)\\
           1, & when \ (x,y) = (0,0)\\
   \end{cases}
\end{equation*}


\vspace{-\baselineskip}
\Heading{Solution:}

\vspace{1ex}
$\begin{aligned}
   &\text{At,\ } (x,y) = (0,0) \quad f(x,y)=1 \quad \text{defined.}\\
   &\text{At,\ } (x,y) \neq (0,0) \quad f(x,y)=\frac{\sin{(x^2+y^2)}}{x^2+y^2}\\
\end{aligned}$

\vspace{2ex}
Now, \ $\Lt f(x, y) \ = \ \Lt \frac{\sin{(x^2+y^2)}}{x^2+y^2} \ = \ 1$

\vspace{3ex}
Also, $\Lt f(x, y) = f(0,0)$\\[1ex]
$\therefore f(x,y)$ is continuous at $(0,0)$.

\vspace{3ex}
\Title{Partial Derivatives}
If $y=f(x)$, then the derivative of $y=f(x)$
with respect to $x$,\\[1ex]
$\frac{dy}{dx} \text{ or } f^\prime(x) = \Lt[0][h] \frac{f(x+h)-f(x)}{h}$ if exists.

\vspace{4ex}
If $u=f(x,y)$ then the partial derivatives, of $u=f(x,y)$ with respect to $x$, keeping $y$ as constant, denoted by $\frac{\partial{y}}{\partial{x}} \text{ or } f_{x}(x,y) = \Lt[0][h] \frac{f(x+h,y)-f(x,y)}{h}$ if exists.

\vspace{5ex}
\textbf{\mred{Exr1.}} Given \ $f(x,y) = x^3+3xy^2+6y^3$. Now, \ find the following,

\tabs \tabs $(i) \ f_x(x,y)$, \quad $(ii) \ f_y(x,y) \quad (iii) \ f_{xy}(x,y)$, \quad $(iv) \ f_{yx}(x,y) \quad (v) \ f_{xx}(x,y)$, \quad $(vi) \ f_{yy}(x,y)$.


\Heading{Solution:}

\vspace{-0.5\baselineskip}
\begin{enumerate}[label=$(\roman*)$]
   \item $f_x(x,y)
   \ = \ \frac{\partial}{\partial{x}}\left[f(x,y)\right]
   \ = \ \frac{\partial}{\partial{x}}\left[x^3+3xy^2+6y^3\right]
   \ = \ 3x^2+3y^2+0
   \ = \ 3x^2+3y^2$
   \vspace{2ex}
   \item $f_y(x,y)
   \ = \ \frac{\partial}{\partial{y}}\left[f(x,y)\right]
   \ = \ \frac{\partial}{\partial{y}}\left[x^3+3xy^2+6y^3\right]
   \ = \ 0+6xy+18y^2
   \ = \ 6xy+18y^2$
   \vspace{2ex}
   \item $f_{xy}(x,y)
   \ = \ \frac{\partial}{\partial{y}}\left[f_x(x,y)\right]
   \ = \ \frac{\partial}{\partial{y}}\left[3x^2+3y^2\right]
   \ = \ 0+6y \ = \ 6y$
   \vspace{2ex}
   \item $f_{yx}(x,y)
   \ = \ \frac{\partial}{\partial{x}}\left[f_y(x,y)\right]
   \ = \ \frac{\partial}{\partial{x}}\left[6xy+18y^2\right]
   \ = \ 6y+0 \ = \ 6y$
   \vspace{2ex}
   \item $f_{xx}(x,y)
   \ = \ \frac{\partial}{\partial{x}}\left[f_x(x,y)\right]
   \ = \ \frac{\partial}{\partial{x}}\left[3x^2+3y^2\right]
   \ = \ 6x+0 \ = \ 6x$
   \vspace{2ex}
   \item $f_{yy}(x,y)
   \ = \ \frac{\partial}{\partial{y}}\left[f_y(x,y)\right]
   \ = \ \frac{\partial}{\partial{y}}\left[6xy+18y^2\right]
   \ = \ 6x+36y$
   \vspace{2ex}
   \item $f_{xx}(1,2)
   \ = \ 6x \ = \ 6(1) \ = \ 6$
   \vspace{2ex}
   \item $f_{yy}(1,2)
   \ = \ 6(1)+36(2) \ = \ 6+72 \ = \ 78$
\end{enumerate}

\vspace{5ex}
\textbf{\mred{Exr2.}} If \ $Z=f(x,y) = \ln{(x^2+y^2)+2\tan^{-1}{\left(\frac{y}{x}\right)}}$. \ Prove that \ $Z_{xx} + Z_{yy} = 0$

\Heading{Solution:}
\vspace{1ex}
$\begin{aligned}
Z_x
\ = \ \frac{\partial}{\partial{x}}[Z]
\ = \ \frac{\partial}{\partial{x}}\left[\ln{(x^2+y^2)+2\tan^{-1}{\left(\frac{y}{x}\right)}}\right]
\ &= \ \frac{2x}{(x^2+y^2)} + 2.\left(\frac{1}{1+\scalebox{1}{$\frac{y^2}{x^2}$}}\right).\left(-\frac{y}{x^2}\right)\\
\ &= \ \frac{2x}{x^2+y^2} - \frac{2y}{x^2+y^2}
\ = \ \frac{2x-2y}{x^2+y^2}
\end{aligned}$

\vspace{3ex}
$\begin{aligned}
Z_y
\ = \ \frac{\partial}{\partial{y}}[Z]
\ = \ \frac{\partial}{\partial{y}}\left[\ln{(x^2+y^2)+2\tan^{-1}{\left(\frac{y}{x}\right)}}\right]
\ &= \ \frac{2y}{(x^2+y^2)} + 2.\left(\frac{1}{1+\scalebox{1}{$\frac{y^2}{x^2}$}}\right).\left(\frac{1}{x}\right)\\
\ &= \ \frac{2y}{x^2+y^2} + \frac{2x}{x^2+y^2}
\ = \ \frac{2x+2y}{x^2+y^2}
\end{aligned}$

\vspace{3ex}
$\begin{aligned}
Z_{xx}
\ = \ \frac{\partial}{\partial{x}}[Z_x]
\ = \ \frac{\partial}{\partial{x}}\left[\frac{2x-2y}{x^2+y^2}\right]
\ = \ \frac{(x^2+y^2)(2)-(2x-2y)(2x)}{(x^2+y^2)^2}
\ = \ \frac{2x^2+2y^2-4x^2+4xy}{(x^2+y^2)^2}
\ = \ \frac{2y^2-2x^2+4xy}{(x^2+y^2)^2}
\end{aligned}$

\vspace{3ex}
$\begin{aligned}
Z_{yy}
\ = \ \frac{\partial}{\partial{y}}[Z_y]
\ = \ \frac{\partial}{\partial{y}}\left[\frac{2x+2y}{x^2+y^2}\right]
\ = \ \frac{(x^2+y^2)(2)-(2x+2y)(2y)}{(x^2+y^2)^2}
\ = \ \frac{2x^2+2y^2-4y^2-4xy}{(x^2+y^2)^2}
\ = \ \frac{2x^2-2y^2-4xy}{(x^2+y^2)^2}
\end{aligned}$

\vspace{5ex}
$\begin{aligned}
   L.H.S \ = \ Z_{xx} + Z_{yy}
   \ = \ \frac{2y^2-2x^2+4xy}{(x^2+y^2)^2}
   + \frac{2x^2-2y^2-4xy}{(x^2+y^2)^2}
   \ = \ \frac{2y^2-2x^2+4xy+2x^2-2y^2-4xy}{(x^2+y^2)^2}
   \ = 0 &= R.H.S\\[-2ex]
   & \text{(Proved)}
\end{aligned}$

\pagebreak
\textbf{\mred{Exr3.}} If the resistors of $R_1$, $R_2$, and $R_3$ ohms are connected in paraller to make an R ohms resistor, the value of $R$ can be found from the equation: $\frac{1}{R} = \frac{1}{R_1}+\frac{1}{R_2}+\frac{1}{R_3}$. 

Find The value of $\frac{\partial{R}}{\partial{R_2}}$
where, $R_1=30$, $R_2=45$, and $R_3=90$ ohms.

\Heading{Solution:}
We have to find \ $\frac{\partial{R}}{\partial{R_2}} = \frac{\partial}{\partial{R_2}}(R)$

\vspace{2ex}
\begin{minipage}[t]{0.49\linewidth}
\noindent
Now,\\[1ex]
$\begin{aligned}
   &\frac{\partial}{\partial{R_2}} \left(\frac{1}{R}\right)
   \ = \ \frac{\partial}{\partial{R_2}} \left(\frac{1}{R_1}+\frac{1}{R_2}+\frac{1}{R_3}\right)\\[1ex]
   &\Rightarrow \ -\frac{1}{R^2} \ \ \frac{\partial{R}}{\partial{R_2}}
   \ = \ 0 - \frac{1}{R_2^2} + 0\\[1ex]
   &\Rightarrow \ \frac{\partial{R}}{\partial{R_2}}
   \ = \ \left(\frac{R}{R_2}\right)^2
   \ = \ \left(\frac{15}{45}\right)^2
   \ = \ \frac{1}{9}
\end{aligned}$
\end{minipage}\hspace{0.5ex}{\vrule width 1pt}\hspace{0.5ex}
\begin{minipage}[t]{0.49\linewidth}
\noindent
Since,\\[1ex]
$\begin{aligned}
   \frac{1}{R} \ &= \frac{1}{R_1}+\frac{1}{R_2}+\frac{1}{R_3}\\[1ex]
   \ &= \ \frac{1}{30}+\frac{1}{45}+\frac{1}{90}
   \ = \ \frac{3+2+1}{90} \ = \ \frac{1}{15}
\end{aligned}$

\vspace{3ex}
$\therefore R = 15$
\end{minipage}

\vspace{5ex}
\Title{Curl, Gradient, Divergence \& Laplacian}
Let $\phi$ be a scalar function and $V$ be a vector, then,

\vspace{1ex}
\textbf{Gradient:} \ \ 
The gradient of $\phi$, denoted by $\operatorname{grad} \phi$ or $\vec{\nabla}\phi$ and defined as

\vspace{0.5ex}
$\operatorname{grad} \phi \ = \ \vec{\nabla}\phi \ = \ \left(i\frac{\partial}{\partial{x}} + j\frac{\partial}{\partial{y}} + k\frac{\partial}{\partial{z}}\right)\phi$


\vspace{2ex}
\textbf{Divergence:} \ \ 
The divergence of $V$, denoted by $\operatorname{div} V$ and defined as

\vspace{0.5ex}
$\begin{aligned}
\operatorname{div } V \ &= \ \vec{\nabla} \cdot \vec{V}
\ \ = \ \ \left(\vec{i} \ \frac{\partial}{\partial x}+\vec{j} \ \frac{\partial}{\partial y}+\vec{k} \ \frac{\partial}{\partial z}\right) \cdot\left(v_1 \ \vec{i}+v_2 \ \vec{j}+v_3 \ \vec{k}\right) 
\ \ = \ \ \frac{\partial v_1}{\partial x}+\frac{\partial v_2}{\partial y}+\frac{\partial v_3}{\partial z} \quad \text { (scaler) }
\end{aligned}$

\vspace{2ex}
\textbf{Carl:} \ \ 
The curl of $V$, denoted by $\operatorname{curl} V$ and defined as

\vspace{0.5ex}
$\begin{aligned}
\operatorname{curl} V & =\vec{\nabla} \times \vec{v}
\ \ = \ \ \left(\vec{i} \ \frac{\partial}{\partial x}+\vec{j} \ \frac{\partial}{\partial y}+\vec{k} \ \frac{\partial}{\partial z}\right) \times\left(v_1 \ \vec{i}+v_2 \ \vec{j}+v_3 \ \vec{k}\right)
\ \ = \ \ \left|\begin{array}{ccc}
\vec{i} & \vec{j} & \vec{k} \\[1ex]
\frac{\partial}{\partial x} & \frac{\partial}{\partial y} & \frac{\partial}{\partial z} \\
v_1 & v_2 & v_3
\end{array}\right|
\end{aligned}$

\textbf{Laplacian:} \ \ 
$\begin{aligned}
& \vec{\nabla} \cdot(\vec{\nabla} \phi)
\ = \ \vec{\nabla} \cdot \left(\vec{i} \frac{\partial \phi}{\partial x}+\vec{j} \frac{\partial \phi}{\partial y}+\vec{k} \frac{\partial \phi}{\partial z}\right)
\ = \ \left(\vec{i} \frac{\partial}{\partial x}+\vec{j} \frac{\partial}{\partial y}+\vec{k} \frac{\partial}{\partial z}\right) \cdot \left(\vec{i} \frac{\partial \phi}{\partial x}+\vec{j} \frac{\partial \phi}{\partial y}+\vec{k} \frac{\partial \phi}{\partial z}\right)
\ = \ \left(\frac{\partial^2}{\partial x^2}+\frac{\partial^2}{\partial y^2}+\frac{\partial^2}{\partial z^2}\right) \phi
\end{aligned}$

\vspace{1ex}
where $\nabla^2=\frac{\partial^2}{\partial x^2}+\frac{\partial^2}{\partial y^2}+\frac{\partial}{\partial z^2}$ is called the laplacian operator.

\pagebreak
\textbf{\mred{Exr-1:}} If $\phi(x, y, z)=3 x^2 y-y^3 z^2$ find the grad $\phi$ at the point $(1,-2,-1)$

\Heading{Solution:}
\vspace{2ex}
$\begin{aligned}
\operatorname{grad} \phi \ &= \
\vec{i} \ \frac{\partial \phi}{\partial x} +
\vec{j} \ \frac{\partial \phi}{\partial y} +
\vec{k} \ \frac{\partial \phi}{\partial z} \\
\ &= \
\vec{i} \ \frac{\partial}{\partial x}\left(3 x^2 y-y^3 z^2\right) + 
\vec{j} \ \frac{\partial}{\partial y}\left(3 x^2 y-y^3 z^2\right) +
\vec{k} \ \frac{\partial}{\partial z}\left(3 x^2 y-y^3 z^2\right) \\
\ &= \
\vec{i} \ (6xy-0) +
\vec{j} \ \left(3x^2-3y^2z^2\right) +
\vec{k} \ \left(0-2y^3z\right)
\end{aligned}$

\vspace{3ex}
At the point \ $(1,-2,-1)$,\\
$\begin{aligned}
\operatorname{grad} \phi
& \ = \ \vec{i} \ \left\{6(1)(-2)-0\right\}+\vec{j} \ \left\{3(1)^2-3(-2)^2(-1)^2\right\}+\vec{k} \ \left\{0-2(-2)^3(-1)\right\}
\ = \ -12\vec{i} -9\vec{j} - 16\vec{k}
\end{aligned}$

\vspace{3ex}
$\begin{aligned}
\text{Now, } \operatorname{curl}\left(\operatorname{grad} \phi\right)
\ &= \ \vec{\nabla} \times \left(\operatorname{grad} \phi\right)\\
\ &= \ \left|\begin{array}{ccc}
   \vec{i} & \vec{j} & \vec{k} \\[1ex]
   \frac{\partial}{\partial x} & \frac{\partial}{\partial y} & \frac{\partial}{\partial z} \\
   6xy & \ \ 3x^2-3y^2z^2 \ \  & -2y^3z
\end{array}\right|\\
\ &= \ \vec{i} \ \left\{-6 y^2 z-0+6 y^2 z\right\}-\vec{j} \ \{0-0\}+\vec{k} \ \{6 x-0-6 x\} \\
\ &= \ 0 \ \vec{i} + 0 \ \vec{j} + 0 \ \vec{k}
\ = \ \vec{0}
\end{aligned}$

\vspace{5ex}
\begin{tabular}{ll}
   \textbf{\mred{Exr-2:}} Prove that & $(i)$ \ the curl of the gradient of scaler function $\phi$ is zero\\
   & $(ii)$ \ the divergence of the carl of a vector $\vec{u}$ is zero(sealer).
\end{tabular}


\Heading{Solution:}
\begin{enumerate}[label=\textbf{\mred{(\roman*)}}]
\item 
Let $\phi$ be a scalar function then $\operatorname{grad} \phi=\vec{\nabla} \phi
\ = \ \vec{i} \frac{\partial \phi}{\partial x}+\vec{j} \frac{\partial \phi}{\partial y}+\vec{k} \frac{\partial \phi}{\partial z}$

\vspace{1ex}
$\begin{aligned}
   \text{Now, } \operatorname{curl}\left(\operatorname{grad} \phi\right)
   \ &= \ \vec{\nabla} \times \left(\vec{\nabla} \phi\right)\\
   \ &= \ \left|\begin{array}{ccc}
      \vec{i} & \vec{j} & \vec{k} \\[1ex]
      \frac{\partial}{\partial x} & \frac{\partial}{\partial y} & \frac{\partial}{\partial z} \\[2ex]
      \frac{\partial{\phi}}{\partial x} & \frac{\partial{\phi}}{\partial y} & \frac{\partial{\phi}}{\partial z} \\
   \end{array}\right|\\[1ex]
   & \ = \ \vec{i}\left\{\frac{\partial^2 \phi}{\partial y \partial z}-\frac{\partial^2 \phi}{\partial z \partial y}\right\}-\vec{j}\left\{\frac{\partial^2 \phi}{\partial x \partial z}-\frac{\partial^2 \phi}{\partial z \partial x}\right\} + \vec{k}\left\{\frac{\partial^2 \phi}{\partial x \partial y}-\frac{\partial^2 \phi}{\partial y \partial x}\right\}
   \\
   & = \ \vec{0}
\end{aligned}$

\pagebreak
\item
Here,\quad
$ \vec{u} \ = \ u_1 \ \vec{i}+u_2 \ \vec{j}+u_3 \ \vec{k}$\\
$\begin{aligned}
   \operatorname{curl} \vec{u} \ = \ \vec{\nabla} \times \vec{u} \ &= \
   \left|\begin{array}{ccc}
      \vec{i} & \vec{j} & \vec{v} \\[1ex]
      \frac{\partial}{\partial x} & \frac{\partial}{\partial y} & \frac{\partial}{\partial z} \\[1ex]
      u_1 & u_2 & u_3
   \end{array}\right| \\[1ex]
   \ &= \ \vec{i}\left\{\frac{\partial u_3}{\partial y}-\frac{\partial u_2}{\partial z}\right\} - \vec{j}\left\{\frac{\partial u_3}{\partial x}-\frac{\partial u_1}{\partial z}\right\} + \vec{k}\left\{\frac{\partial u_2}{\partial x}-\frac{\partial u_1}{\partial y}\right\} \\
\end{aligned}$

Now,\\
$\begin{aligned}
\operatorname{div}\left(\operatorname{curl} \vec{u}\right) \ & = \ \vec{\nabla} \cdot (\operatorname{curl} \vec{u})\\[1ex]
\ &= \ \left(i\frac{\partial}{\partial{x}} + j\frac{\partial}{\partial{y}} + k\frac{\partial}{\partial{z}}\right) \cdot \left(\vec{i}\left\{\frac{\partial u_3}{\partial y}-\frac{\partial u_2}{\partial z}\right\} - \vec{j}\left\{\frac{\partial u_3}{\partial x}-\frac{\partial u_1}{\partial z}\right\} + \vec{k}\left\{\frac{\partial u_2}{\partial x}-\frac{\partial u_1}{\partial y}\right\}\right) \\[2ex]
\ &= \
\frac{\partial}{\partial x}\left[\frac{\partial u_3}{\partial y}-\frac{\partial u_2}{\partial z}\right] 
-\frac{\partial}{\partial y}\left[\frac{\partial u_3}{\partial x}-\frac{\partial u_1}{\partial z}\right] 
+\frac{\partial}{\partial z}\left[\frac{\partial u_2}{\partial x}-\frac{\partial u_1}{\partial y}\right] \\[2ex]
& \ = \
\frac{\partial^2 u_3}{\partial x \partial y} -
\frac{\partial^2 u_2}{\partial x \partial z} -
\frac{\partial^2 u_3}{\partial y \partial x} +
\frac{\partial^2 u_1}{\partial y \partial z} +
\frac{\partial^2 u_2}{\partial z \partial x} -
\frac{\partial^2 u_1}{\partial z \partial y} \\
& = 0 \\
\end{aligned}$
\end{enumerate}

\vspace{8ex}
\textbf{\mred{Exr-3:}} If $\vec{u}=3x^2y \ \vec{i}+5xy^2z \ \vec{j}+xyz^3 \ \vec{k}$ find the divergence of $\vec{u}$ at $(1,2,3)$ and gradient of that divergence.

\Heading{Solution:}
$\begin{aligned}
\operatorname{div} \vec{u}
\ = \ \vec{\nabla} \cdot \vec{u}
\ &= \ \left(i\frac{\partial}{\partial{x}} + j\frac{\partial}{\partial{y}} + k\frac{\partial}{\partial{z}}\right) \cdot \left(3x^2y \ \vec{i}+5xy^2z \ \vec{j}+xyz^3 \ \vec{k}\right) \\[1ex]
&=\frac{\partial}{\partial x}\left(3 x^2 y\right)+\frac{\partial}{\partial y}\left(5 x y^2 z\right)+\frac{\partial}{\partial z}\left(x y z^3\right) \\
&=6 x y+10 x y z+3 x y z^2 \rightarrow \text { scaler quantity }
\end{aligned}$

\vspace{3ex}
At the point \ $(1,2,3)$,\quad 
$\begin{aligned}
   \operatorname{div} \vec{u}
\ = \ (6.1.2) + (10.1.2.3) + (3.1.2.9)
\ = \ 12+60+59 \ = \ 126
\end{aligned}$

\vspace{2ex}
Now, \\
$\begin{aligned}
\operatorname{grad}(\operatorname{div} \vec{u}) \ &= \
\vec{i} \ \frac{\partial}{\partial x}\left(6xy+10xyz+3xyz^2\right)+
\vec{j} \ \frac{\partial}{\partial y}\left(6xy+10xyz+3xyz^2\right)+
\vec{k} \ \frac{\partial}{\partial z}\left(6xy+10xyz+3xyz^2\right)\\
\ &= \ \left(6y+10yz+3yz^2\right) \ \vec{i} +
\left(6x+10xz+3xz^2\right) \ \vec{j} +
\left(0+10xy+6xyz\right) \ \vec{k}
\end{aligned}$


\pagebreak
\begin{tcolorbox}
\begin{center}
   \textbf{Solenoidal vector:} \ If \ $\vec{\nabla} \cdot \vec{A}=0$ \ Then $\vec{A}$ is called solenoidal.
   
   \vspace{1ex}
   \textbf{Irrotational vector:} \ If \ $\vec{\nabla} \times \vec{A}=\vec{0}$ \ Then $\vec{A}$ is called, irrotational.
\end{center}
\end{tcolorbox}


% Directional Derivation $=D \cdot D=\operatorname{grad} \phi \cdot \vec{a}  \quad \leftarrow \text{scaler quantity}$

% \vspace{1ex}
% The component of $\vec{\nabla} \phi$ in the direction of a unit vector $\vec{a}$ is given by \ $\vec{\nabla} \phi \cdot \vec{a}$ \ is called the D.D of $\phi$ in the direction $\vec{a}$.


% \vspace{3ex}
\textbf{\mred{Exr-4:}} Find the directional derivation of $\phi(x, y, z)$ $=4xz^3-3x^2y^2z$ at point $(2,-1,2)$ in the direction $2 \vec{i}-3 \vec{j}+6 \vec{k}$



\vspace{3ex}
Here,\\
$\begin{aligned}
\vec{\nabla} \phi \ &= \
\vec{i} \ \frac{\partial}{\partial x}\left(4xz^3-3x^2y^2z\right) +
\vec{j} \ \frac{\partial}{\partial y}\left(4xz^3-3x^2y^2z\right) +
\vec{k} \ \frac{\partial}{\partial z}\left(4xz^3-3x^2y^2z\right) \\
\ &= \ \left(4 z^3-6 x y^2 z\right) \vec{i}+\left(0-6 x^2 y z\right) \vec{j}+\left(12 x z^2-3 x^2 y^2\right) \vec{k} \\
\end{aligned}$

\vspace{3ex}
At $(2,-1,2)$, \quad 
$\begin{aligned}
\vec{\nabla} \phi \ &= \ (32-24) \vec{i}- \left(0 - 6.4.(-1).2\right) \vec{j}+(96-12) \vec{k}
\ = \ 8 \vec{i}+48 \vec{j}+84 \vec{k}
\end{aligned}$

\vspace{2ex}
$\therefore \vec{a} \ = \ \frac{2 i+3 \vec{j}+6 \vec{k}}{\sqrt{4+9+36}} \ = \ =\frac{2}{7} \vec{i} - \frac{3}{7} \vec{j} + \frac{6}{7} \vec{k}$

\vspace{3ex}
Now,

$\begin{aligned}
   \text{Directional Derivation } D.D \ &= \
   \vec{\nabla} \phi \cdot \vec{a}\\[1ex]
   &= \left(8 \vec{i}+48 \vec{j}+84 \vec{k}\right) \cdot \left(\frac{2}{7} \vec{i} - \frac{3}{7} \vec{j} + \frac{6}{7} \vec{k}\right)\\[1ex]
   &= \ 8 \cdot \frac{2}{7} + 48 \cdot \frac{-3}{7} + 84 \cdot \frac{6}{7} \quad = \ \frac{16-144+504}{7} \quad = \ \frac{376}{7}
\end{aligned}$
\end{document}